\section{Why Does Image Clip perform better than Video Clip?}
\label{sec:ablation_vc}
\subsection{Domain Adaptation Gap}
VideoCLIP is pretrained on the k400 dataset \parencite{kay2017kinetics}, which is a human action dataset, while ImageCLIP is pretrained on a combination of web-crawled and commonly used pre-existing image datasets. As suggested in \parencite{farahani2021brief}, the gap between the animal and human domains may lead to poor performance.

\subsection{Need for More Learnable Weights}
With the Animal Kingdom dataset containing more than 50 hours of animal video clips, it may require more trainable weights to achieve a better fit. To test the effect of different sizes of trainable weights, the following two settings are experimented with to add trainable weights:

\begin{enumerate}
    \item VC\_AT: The VC\_Proj model adds a 12-layer transformer, which is the same as the Image Clip settings in Figure \ref{fig:modelstructure_ic}. Rather than pooling all outputs of ViT's last layer, the embeddings of patches in the same frame are pooled together to obtain the frame-based embedding, which is used as input for the 12-layer transformer.
    \item VC\_DF: This setting is the same as the VC\_Proj model but with two more learnable layers. The Uniformer model, which is the structure used by VideoCLIP, enhances the cross-frame relationship on the final 4 layers of ViT. There are two mechanisms to achieve this, Deep Position Embedding (DPE), a series of 3D CNN layers, and Feed Forward Network (FFN), the attention mechanism on the cls token with two linear layers. These two layers are learnable in this setting.
\end{enumerate}

Table \ref{tab:ablation_vc} and Figure \ref{fig:ablation_vc} show the results of the models with different learnable weights and the performance on each epoch. As illustrated in the figure, it is clear that the VC\_AT model is able to improve the VC\_Proj model to get closer to IC, proving that more learnable weights are indeed helpful for better fitting. Taking advantage of the well-designed mechanism of Uniformer, the VC\_DF model is able to achieve a higher score (55.98 mAP) than the IC model (54.36) as demonstrated in the table. 

\begin{table}[ht]
    \centering
    \caption{Training Results for Models with Different Learnable Weights}
    \label{tab:ablation_vc}
    \begin{tabular}{lllll}
        \toprule
        \multirow{2}{*}{Models} & Accuracy \\
        \cmidrule{2-2} 
        {} &  Best Epoch \\
        \midrule
        VC\_Vision & 25.74 \\
        VC\_Proj   & 48.56 \\
        IC         & 54.36 \\
        VC\_AT     & 52.79 \\
        VC\_DF     & 55.98 \\
        \bottomrule
    \end{tabular}
\end{table}

\begin{figure}[ht]
    \centering
    \resizebox{1.0\textwidth}{!}{%% Creator: Matplotlib, PGF backend
%%
%% To include the figure in your LaTeX document, write
%%   \input{<filename>.pgf}
%%
%% Make sure the required packages are loaded in your preamble
%%   \usepackage{pgf}
%%
%% and, on pdftex
%%   \usepackage[utf8]{inputenc}\DeclareUnicodeCharacter{2212}{-}
%%
%% or, on luatex and xetex
%%   \usepackage{unicode-math}
%%
%% Figures using additional raster images can only be included by \input if
%% they are in the same directory as the main LaTeX file. For loading figures
%% from other directories you can use the `import` package
%%   \usepackage{import}
%%
%% and then include the figures with
%%   \import{<path to file>}{<filename>.pgf}
%%
%% Matplotlib used the following preamble
%%
\begingroup%
\makeatletter%
\begin{pgfpicture}%
\pgfpathrectangle{\pgfpointorigin}{\pgfqpoint{7.000000in}{4.000000in}}%
\pgfusepath{use as bounding box, clip}%
\begin{pgfscope}%
\pgfsetbuttcap%
\pgfsetmiterjoin%
\definecolor{currentfill}{rgb}{1.000000,1.000000,1.000000}%
\pgfsetfillcolor{currentfill}%
\pgfsetlinewidth{0.000000pt}%
\definecolor{currentstroke}{rgb}{1.000000,1.000000,1.000000}%
\pgfsetstrokecolor{currentstroke}%
\pgfsetdash{}{0pt}%
\pgfpathmoveto{\pgfqpoint{0.000000in}{0.000000in}}%
\pgfpathlineto{\pgfqpoint{7.000000in}{0.000000in}}%
\pgfpathlineto{\pgfqpoint{7.000000in}{4.000000in}}%
\pgfpathlineto{\pgfqpoint{0.000000in}{4.000000in}}%
\pgfpathclose%
\pgfusepath{fill}%
\end{pgfscope}%
\begin{pgfscope}%
\pgfsetbuttcap%
\pgfsetmiterjoin%
\definecolor{currentfill}{rgb}{1.000000,1.000000,1.000000}%
\pgfsetfillcolor{currentfill}%
\pgfsetlinewidth{0.000000pt}%
\definecolor{currentstroke}{rgb}{0.000000,0.000000,0.000000}%
\pgfsetstrokecolor{currentstroke}%
\pgfsetstrokeopacity{0.000000}%
\pgfsetdash{}{0pt}%
\pgfpathmoveto{\pgfqpoint{0.875000in}{0.440000in}}%
\pgfpathlineto{\pgfqpoint{6.300000in}{0.440000in}}%
\pgfpathlineto{\pgfqpoint{6.300000in}{3.520000in}}%
\pgfpathlineto{\pgfqpoint{0.875000in}{3.520000in}}%
\pgfpathclose%
\pgfusepath{fill}%
\end{pgfscope}%
\begin{pgfscope}%
\pgfpathrectangle{\pgfqpoint{0.875000in}{0.440000in}}{\pgfqpoint{5.425000in}{3.080000in}}%
\pgfusepath{clip}%
\pgfsetrectcap%
\pgfsetroundjoin%
\pgfsetlinewidth{0.803000pt}%
\definecolor{currentstroke}{rgb}{0.690196,0.690196,0.690196}%
\pgfsetstrokecolor{currentstroke}%
\pgfsetdash{}{0pt}%
\pgfpathmoveto{\pgfqpoint{1.121591in}{0.440000in}}%
\pgfpathlineto{\pgfqpoint{1.121591in}{3.520000in}}%
\pgfusepath{stroke}%
\end{pgfscope}%
\begin{pgfscope}%
\pgfsetbuttcap%
\pgfsetroundjoin%
\definecolor{currentfill}{rgb}{0.000000,0.000000,0.000000}%
\pgfsetfillcolor{currentfill}%
\pgfsetlinewidth{0.803000pt}%
\definecolor{currentstroke}{rgb}{0.000000,0.000000,0.000000}%
\pgfsetstrokecolor{currentstroke}%
\pgfsetdash{}{0pt}%
\pgfsys@defobject{currentmarker}{\pgfqpoint{0.000000in}{-0.048611in}}{\pgfqpoint{0.000000in}{0.000000in}}{%
\pgfpathmoveto{\pgfqpoint{0.000000in}{0.000000in}}%
\pgfpathlineto{\pgfqpoint{0.000000in}{-0.048611in}}%
\pgfusepath{stroke,fill}%
}%
\begin{pgfscope}%
\pgfsys@transformshift{1.121591in}{0.440000in}%
\pgfsys@useobject{currentmarker}{}%
\end{pgfscope}%
\end{pgfscope}%
\begin{pgfscope}%
\definecolor{textcolor}{rgb}{0.000000,0.000000,0.000000}%
\pgfsetstrokecolor{textcolor}%
\pgfsetfillcolor{textcolor}%
\pgftext[x=1.121591in,y=0.342778in,,top]{\color{textcolor}\rmfamily\fontsize{10.000000}{12.000000}\selectfont \(\displaystyle {0}\)}%
\end{pgfscope}%
\begin{pgfscope}%
\pgfpathrectangle{\pgfqpoint{0.875000in}{0.440000in}}{\pgfqpoint{5.425000in}{3.080000in}}%
\pgfusepath{clip}%
\pgfsetrectcap%
\pgfsetroundjoin%
\pgfsetlinewidth{0.803000pt}%
\definecolor{currentstroke}{rgb}{0.690196,0.690196,0.690196}%
\pgfsetstrokecolor{currentstroke}%
\pgfsetdash{}{0pt}%
\pgfpathmoveto{\pgfqpoint{2.242459in}{0.440000in}}%
\pgfpathlineto{\pgfqpoint{2.242459in}{3.520000in}}%
\pgfusepath{stroke}%
\end{pgfscope}%
\begin{pgfscope}%
\pgfsetbuttcap%
\pgfsetroundjoin%
\definecolor{currentfill}{rgb}{0.000000,0.000000,0.000000}%
\pgfsetfillcolor{currentfill}%
\pgfsetlinewidth{0.803000pt}%
\definecolor{currentstroke}{rgb}{0.000000,0.000000,0.000000}%
\pgfsetstrokecolor{currentstroke}%
\pgfsetdash{}{0pt}%
\pgfsys@defobject{currentmarker}{\pgfqpoint{0.000000in}{-0.048611in}}{\pgfqpoint{0.000000in}{0.000000in}}{%
\pgfpathmoveto{\pgfqpoint{0.000000in}{0.000000in}}%
\pgfpathlineto{\pgfqpoint{0.000000in}{-0.048611in}}%
\pgfusepath{stroke,fill}%
}%
\begin{pgfscope}%
\pgfsys@transformshift{2.242459in}{0.440000in}%
\pgfsys@useobject{currentmarker}{}%
\end{pgfscope}%
\end{pgfscope}%
\begin{pgfscope}%
\definecolor{textcolor}{rgb}{0.000000,0.000000,0.000000}%
\pgfsetstrokecolor{textcolor}%
\pgfsetfillcolor{textcolor}%
\pgftext[x=2.242459in,y=0.342778in,,top]{\color{textcolor}\rmfamily\fontsize{10.000000}{12.000000}\selectfont \(\displaystyle {10}\)}%
\end{pgfscope}%
\begin{pgfscope}%
\pgfpathrectangle{\pgfqpoint{0.875000in}{0.440000in}}{\pgfqpoint{5.425000in}{3.080000in}}%
\pgfusepath{clip}%
\pgfsetrectcap%
\pgfsetroundjoin%
\pgfsetlinewidth{0.803000pt}%
\definecolor{currentstroke}{rgb}{0.690196,0.690196,0.690196}%
\pgfsetstrokecolor{currentstroke}%
\pgfsetdash{}{0pt}%
\pgfpathmoveto{\pgfqpoint{3.363326in}{0.440000in}}%
\pgfpathlineto{\pgfqpoint{3.363326in}{3.520000in}}%
\pgfusepath{stroke}%
\end{pgfscope}%
\begin{pgfscope}%
\pgfsetbuttcap%
\pgfsetroundjoin%
\definecolor{currentfill}{rgb}{0.000000,0.000000,0.000000}%
\pgfsetfillcolor{currentfill}%
\pgfsetlinewidth{0.803000pt}%
\definecolor{currentstroke}{rgb}{0.000000,0.000000,0.000000}%
\pgfsetstrokecolor{currentstroke}%
\pgfsetdash{}{0pt}%
\pgfsys@defobject{currentmarker}{\pgfqpoint{0.000000in}{-0.048611in}}{\pgfqpoint{0.000000in}{0.000000in}}{%
\pgfpathmoveto{\pgfqpoint{0.000000in}{0.000000in}}%
\pgfpathlineto{\pgfqpoint{0.000000in}{-0.048611in}}%
\pgfusepath{stroke,fill}%
}%
\begin{pgfscope}%
\pgfsys@transformshift{3.363326in}{0.440000in}%
\pgfsys@useobject{currentmarker}{}%
\end{pgfscope}%
\end{pgfscope}%
\begin{pgfscope}%
\definecolor{textcolor}{rgb}{0.000000,0.000000,0.000000}%
\pgfsetstrokecolor{textcolor}%
\pgfsetfillcolor{textcolor}%
\pgftext[x=3.363326in,y=0.342778in,,top]{\color{textcolor}\rmfamily\fontsize{10.000000}{12.000000}\selectfont \(\displaystyle {20}\)}%
\end{pgfscope}%
\begin{pgfscope}%
\pgfpathrectangle{\pgfqpoint{0.875000in}{0.440000in}}{\pgfqpoint{5.425000in}{3.080000in}}%
\pgfusepath{clip}%
\pgfsetrectcap%
\pgfsetroundjoin%
\pgfsetlinewidth{0.803000pt}%
\definecolor{currentstroke}{rgb}{0.690196,0.690196,0.690196}%
\pgfsetstrokecolor{currentstroke}%
\pgfsetdash{}{0pt}%
\pgfpathmoveto{\pgfqpoint{4.484194in}{0.440000in}}%
\pgfpathlineto{\pgfqpoint{4.484194in}{3.520000in}}%
\pgfusepath{stroke}%
\end{pgfscope}%
\begin{pgfscope}%
\pgfsetbuttcap%
\pgfsetroundjoin%
\definecolor{currentfill}{rgb}{0.000000,0.000000,0.000000}%
\pgfsetfillcolor{currentfill}%
\pgfsetlinewidth{0.803000pt}%
\definecolor{currentstroke}{rgb}{0.000000,0.000000,0.000000}%
\pgfsetstrokecolor{currentstroke}%
\pgfsetdash{}{0pt}%
\pgfsys@defobject{currentmarker}{\pgfqpoint{0.000000in}{-0.048611in}}{\pgfqpoint{0.000000in}{0.000000in}}{%
\pgfpathmoveto{\pgfqpoint{0.000000in}{0.000000in}}%
\pgfpathlineto{\pgfqpoint{0.000000in}{-0.048611in}}%
\pgfusepath{stroke,fill}%
}%
\begin{pgfscope}%
\pgfsys@transformshift{4.484194in}{0.440000in}%
\pgfsys@useobject{currentmarker}{}%
\end{pgfscope}%
\end{pgfscope}%
\begin{pgfscope}%
\definecolor{textcolor}{rgb}{0.000000,0.000000,0.000000}%
\pgfsetstrokecolor{textcolor}%
\pgfsetfillcolor{textcolor}%
\pgftext[x=4.484194in,y=0.342778in,,top]{\color{textcolor}\rmfamily\fontsize{10.000000}{12.000000}\selectfont \(\displaystyle {30}\)}%
\end{pgfscope}%
\begin{pgfscope}%
\pgfpathrectangle{\pgfqpoint{0.875000in}{0.440000in}}{\pgfqpoint{5.425000in}{3.080000in}}%
\pgfusepath{clip}%
\pgfsetrectcap%
\pgfsetroundjoin%
\pgfsetlinewidth{0.803000pt}%
\definecolor{currentstroke}{rgb}{0.690196,0.690196,0.690196}%
\pgfsetstrokecolor{currentstroke}%
\pgfsetdash{}{0pt}%
\pgfpathmoveto{\pgfqpoint{5.605062in}{0.440000in}}%
\pgfpathlineto{\pgfqpoint{5.605062in}{3.520000in}}%
\pgfusepath{stroke}%
\end{pgfscope}%
\begin{pgfscope}%
\pgfsetbuttcap%
\pgfsetroundjoin%
\definecolor{currentfill}{rgb}{0.000000,0.000000,0.000000}%
\pgfsetfillcolor{currentfill}%
\pgfsetlinewidth{0.803000pt}%
\definecolor{currentstroke}{rgb}{0.000000,0.000000,0.000000}%
\pgfsetstrokecolor{currentstroke}%
\pgfsetdash{}{0pt}%
\pgfsys@defobject{currentmarker}{\pgfqpoint{0.000000in}{-0.048611in}}{\pgfqpoint{0.000000in}{0.000000in}}{%
\pgfpathmoveto{\pgfqpoint{0.000000in}{0.000000in}}%
\pgfpathlineto{\pgfqpoint{0.000000in}{-0.048611in}}%
\pgfusepath{stroke,fill}%
}%
\begin{pgfscope}%
\pgfsys@transformshift{5.605062in}{0.440000in}%
\pgfsys@useobject{currentmarker}{}%
\end{pgfscope}%
\end{pgfscope}%
\begin{pgfscope}%
\definecolor{textcolor}{rgb}{0.000000,0.000000,0.000000}%
\pgfsetstrokecolor{textcolor}%
\pgfsetfillcolor{textcolor}%
\pgftext[x=5.605062in,y=0.342778in,,top]{\color{textcolor}\rmfamily\fontsize{10.000000}{12.000000}\selectfont \(\displaystyle {40}\)}%
\end{pgfscope}%
\begin{pgfscope}%
\definecolor{textcolor}{rgb}{0.000000,0.000000,0.000000}%
\pgfsetstrokecolor{textcolor}%
\pgfsetfillcolor{textcolor}%
\pgftext[x=3.587500in,y=0.163766in,,top]{\color{textcolor}\rmfamily\fontsize{10.000000}{12.000000}\selectfont Epoch}%
\end{pgfscope}%
\begin{pgfscope}%
\pgfpathrectangle{\pgfqpoint{0.875000in}{0.440000in}}{\pgfqpoint{5.425000in}{3.080000in}}%
\pgfusepath{clip}%
\pgfsetrectcap%
\pgfsetroundjoin%
\pgfsetlinewidth{0.803000pt}%
\definecolor{currentstroke}{rgb}{0.690196,0.690196,0.690196}%
\pgfsetstrokecolor{currentstroke}%
\pgfsetdash{}{0pt}%
\pgfpathmoveto{\pgfqpoint{0.875000in}{0.952019in}}%
\pgfpathlineto{\pgfqpoint{6.300000in}{0.952019in}}%
\pgfusepath{stroke}%
\end{pgfscope}%
\begin{pgfscope}%
\pgfsetbuttcap%
\pgfsetroundjoin%
\definecolor{currentfill}{rgb}{0.000000,0.000000,0.000000}%
\pgfsetfillcolor{currentfill}%
\pgfsetlinewidth{0.803000pt}%
\definecolor{currentstroke}{rgb}{0.000000,0.000000,0.000000}%
\pgfsetstrokecolor{currentstroke}%
\pgfsetdash{}{0pt}%
\pgfsys@defobject{currentmarker}{\pgfqpoint{-0.048611in}{0.000000in}}{\pgfqpoint{0.000000in}{0.000000in}}{%
\pgfpathmoveto{\pgfqpoint{0.000000in}{0.000000in}}%
\pgfpathlineto{\pgfqpoint{-0.048611in}{0.000000in}}%
\pgfusepath{stroke,fill}%
}%
\begin{pgfscope}%
\pgfsys@transformshift{0.875000in}{0.952019in}%
\pgfsys@useobject{currentmarker}{}%
\end{pgfscope}%
\end{pgfscope}%
\begin{pgfscope}%
\definecolor{textcolor}{rgb}{0.000000,0.000000,0.000000}%
\pgfsetstrokecolor{textcolor}%
\pgfsetfillcolor{textcolor}%
\pgftext[x=0.600308in, y=0.903794in, left, base]{\color{textcolor}\rmfamily\fontsize{10.000000}{12.000000}\selectfont \(\displaystyle {0.1}\)}%
\end{pgfscope}%
\begin{pgfscope}%
\pgfpathrectangle{\pgfqpoint{0.875000in}{0.440000in}}{\pgfqpoint{5.425000in}{3.080000in}}%
\pgfusepath{clip}%
\pgfsetrectcap%
\pgfsetroundjoin%
\pgfsetlinewidth{0.803000pt}%
\definecolor{currentstroke}{rgb}{0.690196,0.690196,0.690196}%
\pgfsetstrokecolor{currentstroke}%
\pgfsetdash{}{0pt}%
\pgfpathmoveto{\pgfqpoint{0.875000in}{1.480017in}}%
\pgfpathlineto{\pgfqpoint{6.300000in}{1.480017in}}%
\pgfusepath{stroke}%
\end{pgfscope}%
\begin{pgfscope}%
\pgfsetbuttcap%
\pgfsetroundjoin%
\definecolor{currentfill}{rgb}{0.000000,0.000000,0.000000}%
\pgfsetfillcolor{currentfill}%
\pgfsetlinewidth{0.803000pt}%
\definecolor{currentstroke}{rgb}{0.000000,0.000000,0.000000}%
\pgfsetstrokecolor{currentstroke}%
\pgfsetdash{}{0pt}%
\pgfsys@defobject{currentmarker}{\pgfqpoint{-0.048611in}{0.000000in}}{\pgfqpoint{0.000000in}{0.000000in}}{%
\pgfpathmoveto{\pgfqpoint{0.000000in}{0.000000in}}%
\pgfpathlineto{\pgfqpoint{-0.048611in}{0.000000in}}%
\pgfusepath{stroke,fill}%
}%
\begin{pgfscope}%
\pgfsys@transformshift{0.875000in}{1.480017in}%
\pgfsys@useobject{currentmarker}{}%
\end{pgfscope}%
\end{pgfscope}%
\begin{pgfscope}%
\definecolor{textcolor}{rgb}{0.000000,0.000000,0.000000}%
\pgfsetstrokecolor{textcolor}%
\pgfsetfillcolor{textcolor}%
\pgftext[x=0.600308in, y=1.431792in, left, base]{\color{textcolor}\rmfamily\fontsize{10.000000}{12.000000}\selectfont \(\displaystyle {0.2}\)}%
\end{pgfscope}%
\begin{pgfscope}%
\pgfpathrectangle{\pgfqpoint{0.875000in}{0.440000in}}{\pgfqpoint{5.425000in}{3.080000in}}%
\pgfusepath{clip}%
\pgfsetrectcap%
\pgfsetroundjoin%
\pgfsetlinewidth{0.803000pt}%
\definecolor{currentstroke}{rgb}{0.690196,0.690196,0.690196}%
\pgfsetstrokecolor{currentstroke}%
\pgfsetdash{}{0pt}%
\pgfpathmoveto{\pgfqpoint{0.875000in}{2.008016in}}%
\pgfpathlineto{\pgfqpoint{6.300000in}{2.008016in}}%
\pgfusepath{stroke}%
\end{pgfscope}%
\begin{pgfscope}%
\pgfsetbuttcap%
\pgfsetroundjoin%
\definecolor{currentfill}{rgb}{0.000000,0.000000,0.000000}%
\pgfsetfillcolor{currentfill}%
\pgfsetlinewidth{0.803000pt}%
\definecolor{currentstroke}{rgb}{0.000000,0.000000,0.000000}%
\pgfsetstrokecolor{currentstroke}%
\pgfsetdash{}{0pt}%
\pgfsys@defobject{currentmarker}{\pgfqpoint{-0.048611in}{0.000000in}}{\pgfqpoint{0.000000in}{0.000000in}}{%
\pgfpathmoveto{\pgfqpoint{0.000000in}{0.000000in}}%
\pgfpathlineto{\pgfqpoint{-0.048611in}{0.000000in}}%
\pgfusepath{stroke,fill}%
}%
\begin{pgfscope}%
\pgfsys@transformshift{0.875000in}{2.008016in}%
\pgfsys@useobject{currentmarker}{}%
\end{pgfscope}%
\end{pgfscope}%
\begin{pgfscope}%
\definecolor{textcolor}{rgb}{0.000000,0.000000,0.000000}%
\pgfsetstrokecolor{textcolor}%
\pgfsetfillcolor{textcolor}%
\pgftext[x=0.600308in, y=1.959791in, left, base]{\color{textcolor}\rmfamily\fontsize{10.000000}{12.000000}\selectfont \(\displaystyle {0.3}\)}%
\end{pgfscope}%
\begin{pgfscope}%
\pgfpathrectangle{\pgfqpoint{0.875000in}{0.440000in}}{\pgfqpoint{5.425000in}{3.080000in}}%
\pgfusepath{clip}%
\pgfsetrectcap%
\pgfsetroundjoin%
\pgfsetlinewidth{0.803000pt}%
\definecolor{currentstroke}{rgb}{0.690196,0.690196,0.690196}%
\pgfsetstrokecolor{currentstroke}%
\pgfsetdash{}{0pt}%
\pgfpathmoveto{\pgfqpoint{0.875000in}{2.536014in}}%
\pgfpathlineto{\pgfqpoint{6.300000in}{2.536014in}}%
\pgfusepath{stroke}%
\end{pgfscope}%
\begin{pgfscope}%
\pgfsetbuttcap%
\pgfsetroundjoin%
\definecolor{currentfill}{rgb}{0.000000,0.000000,0.000000}%
\pgfsetfillcolor{currentfill}%
\pgfsetlinewidth{0.803000pt}%
\definecolor{currentstroke}{rgb}{0.000000,0.000000,0.000000}%
\pgfsetstrokecolor{currentstroke}%
\pgfsetdash{}{0pt}%
\pgfsys@defobject{currentmarker}{\pgfqpoint{-0.048611in}{0.000000in}}{\pgfqpoint{0.000000in}{0.000000in}}{%
\pgfpathmoveto{\pgfqpoint{0.000000in}{0.000000in}}%
\pgfpathlineto{\pgfqpoint{-0.048611in}{0.000000in}}%
\pgfusepath{stroke,fill}%
}%
\begin{pgfscope}%
\pgfsys@transformshift{0.875000in}{2.536014in}%
\pgfsys@useobject{currentmarker}{}%
\end{pgfscope}%
\end{pgfscope}%
\begin{pgfscope}%
\definecolor{textcolor}{rgb}{0.000000,0.000000,0.000000}%
\pgfsetstrokecolor{textcolor}%
\pgfsetfillcolor{textcolor}%
\pgftext[x=0.600308in, y=2.487789in, left, base]{\color{textcolor}\rmfamily\fontsize{10.000000}{12.000000}\selectfont \(\displaystyle {0.4}\)}%
\end{pgfscope}%
\begin{pgfscope}%
\pgfpathrectangle{\pgfqpoint{0.875000in}{0.440000in}}{\pgfqpoint{5.425000in}{3.080000in}}%
\pgfusepath{clip}%
\pgfsetrectcap%
\pgfsetroundjoin%
\pgfsetlinewidth{0.803000pt}%
\definecolor{currentstroke}{rgb}{0.690196,0.690196,0.690196}%
\pgfsetstrokecolor{currentstroke}%
\pgfsetdash{}{0pt}%
\pgfpathmoveto{\pgfqpoint{0.875000in}{3.064013in}}%
\pgfpathlineto{\pgfqpoint{6.300000in}{3.064013in}}%
\pgfusepath{stroke}%
\end{pgfscope}%
\begin{pgfscope}%
\pgfsetbuttcap%
\pgfsetroundjoin%
\definecolor{currentfill}{rgb}{0.000000,0.000000,0.000000}%
\pgfsetfillcolor{currentfill}%
\pgfsetlinewidth{0.803000pt}%
\definecolor{currentstroke}{rgb}{0.000000,0.000000,0.000000}%
\pgfsetstrokecolor{currentstroke}%
\pgfsetdash{}{0pt}%
\pgfsys@defobject{currentmarker}{\pgfqpoint{-0.048611in}{0.000000in}}{\pgfqpoint{0.000000in}{0.000000in}}{%
\pgfpathmoveto{\pgfqpoint{0.000000in}{0.000000in}}%
\pgfpathlineto{\pgfqpoint{-0.048611in}{0.000000in}}%
\pgfusepath{stroke,fill}%
}%
\begin{pgfscope}%
\pgfsys@transformshift{0.875000in}{3.064013in}%
\pgfsys@useobject{currentmarker}{}%
\end{pgfscope}%
\end{pgfscope}%
\begin{pgfscope}%
\definecolor{textcolor}{rgb}{0.000000,0.000000,0.000000}%
\pgfsetstrokecolor{textcolor}%
\pgfsetfillcolor{textcolor}%
\pgftext[x=0.600308in, y=3.015788in, left, base]{\color{textcolor}\rmfamily\fontsize{10.000000}{12.000000}\selectfont \(\displaystyle {0.5}\)}%
\end{pgfscope}%
\begin{pgfscope}%
\definecolor{textcolor}{rgb}{0.000000,0.000000,0.000000}%
\pgfsetstrokecolor{textcolor}%
\pgfsetfillcolor{textcolor}%
\pgftext[x=0.544752in,y=1.980000in,,bottom,rotate=90.000000]{\color{textcolor}\rmfamily\fontsize{10.000000}{12.000000}\selectfont mAP}%
\end{pgfscope}%
\begin{pgfscope}%
\pgfpathrectangle{\pgfqpoint{0.875000in}{0.440000in}}{\pgfqpoint{5.425000in}{3.080000in}}%
\pgfusepath{clip}%
\pgfsetrectcap%
\pgfsetroundjoin%
\pgfsetlinewidth{1.505625pt}%
\definecolor{currentstroke}{rgb}{0.121569,0.466667,0.705882}%
\pgfsetstrokecolor{currentstroke}%
\pgfsetdash{}{0pt}%
\pgfpathmoveto{\pgfqpoint{1.121591in}{0.794420in}}%
\pgfpathlineto{\pgfqpoint{1.233678in}{1.027653in}}%
\pgfpathlineto{\pgfqpoint{1.345764in}{1.091314in}}%
\pgfpathlineto{\pgfqpoint{1.457851in}{1.196405in}}%
\pgfpathlineto{\pgfqpoint{1.569938in}{1.238666in}}%
\pgfpathlineto{\pgfqpoint{1.682025in}{1.309640in}}%
\pgfpathlineto{\pgfqpoint{1.794112in}{1.325704in}}%
\pgfpathlineto{\pgfqpoint{1.906198in}{1.370793in}}%
\pgfpathlineto{\pgfqpoint{2.018285in}{1.408075in}}%
\pgfpathlineto{\pgfqpoint{2.130372in}{1.434875in}}%
\pgfpathlineto{\pgfqpoint{2.242459in}{1.405676in}}%
\pgfpathlineto{\pgfqpoint{2.354545in}{1.457186in}}%
\pgfpathlineto{\pgfqpoint{2.466632in}{1.469218in}}%
\pgfpathlineto{\pgfqpoint{2.578719in}{1.452434in}}%
\pgfpathlineto{\pgfqpoint{2.690806in}{1.504031in}}%
\pgfpathlineto{\pgfqpoint{2.802893in}{1.495361in}}%
\pgfpathlineto{\pgfqpoint{2.914979in}{1.594839in}}%
\pgfpathlineto{\pgfqpoint{3.027066in}{1.567836in}}%
\pgfpathlineto{\pgfqpoint{3.139153in}{1.592304in}}%
\pgfpathlineto{\pgfqpoint{3.251240in}{1.619042in}}%
\pgfpathlineto{\pgfqpoint{3.363326in}{1.579572in}}%
\pgfpathlineto{\pgfqpoint{3.475413in}{1.618569in}}%
\pgfpathlineto{\pgfqpoint{3.587500in}{1.621920in}}%
\pgfpathlineto{\pgfqpoint{3.699587in}{1.669461in}}%
\pgfpathlineto{\pgfqpoint{3.811674in}{1.678185in}}%
\pgfpathlineto{\pgfqpoint{3.923760in}{1.625954in}}%
\pgfpathlineto{\pgfqpoint{4.035847in}{1.627907in}}%
\pgfpathlineto{\pgfqpoint{4.147934in}{1.660259in}}%
\pgfpathlineto{\pgfqpoint{4.260021in}{1.701571in}}%
\pgfpathlineto{\pgfqpoint{4.372107in}{1.611735in}}%
\pgfpathlineto{\pgfqpoint{4.484194in}{1.620648in}}%
\pgfpathlineto{\pgfqpoint{4.596281in}{1.703444in}}%
\pgfpathlineto{\pgfqpoint{4.708368in}{1.748424in}}%
\pgfpathlineto{\pgfqpoint{4.820455in}{1.653477in}}%
\pgfpathlineto{\pgfqpoint{4.932541in}{1.659607in}}%
\pgfpathlineto{\pgfqpoint{5.044628in}{1.731411in}}%
\pgfpathlineto{\pgfqpoint{5.156715in}{1.758110in}}%
\pgfpathlineto{\pgfqpoint{5.268802in}{1.674757in}}%
\pgfpathlineto{\pgfqpoint{5.380888in}{1.675480in}}%
\pgfpathlineto{\pgfqpoint{5.492975in}{1.740303in}}%
\pgfpathlineto{\pgfqpoint{5.605062in}{1.773579in}}%
\pgfpathlineto{\pgfqpoint{5.717149in}{1.705983in}}%
\pgfpathlineto{\pgfqpoint{5.829236in}{1.778915in}}%
\pgfpathlineto{\pgfqpoint{5.941322in}{1.783211in}}%
\pgfpathlineto{\pgfqpoint{6.053409in}{1.722756in}}%
\pgfusepath{stroke}%
\end{pgfscope}%
\begin{pgfscope}%
\pgfpathrectangle{\pgfqpoint{0.875000in}{0.440000in}}{\pgfqpoint{5.425000in}{3.080000in}}%
\pgfusepath{clip}%
\pgfsetrectcap%
\pgfsetroundjoin%
\pgfsetlinewidth{1.505625pt}%
\definecolor{currentstroke}{rgb}{1.000000,0.498039,0.054902}%
\pgfsetstrokecolor{currentstroke}%
\pgfsetdash{}{0pt}%
\pgfpathmoveto{\pgfqpoint{1.121591in}{0.721853in}}%
\pgfpathlineto{\pgfqpoint{1.233678in}{0.989522in}}%
\pgfpathlineto{\pgfqpoint{1.345764in}{1.316511in}}%
\pgfpathlineto{\pgfqpoint{1.457851in}{1.566373in}}%
\pgfpathlineto{\pgfqpoint{1.569938in}{1.767084in}}%
\pgfpathlineto{\pgfqpoint{1.682025in}{1.922472in}}%
\pgfpathlineto{\pgfqpoint{1.794112in}{2.059887in}}%
\pgfpathlineto{\pgfqpoint{1.906198in}{2.159993in}}%
\pgfpathlineto{\pgfqpoint{2.018285in}{2.159395in}}%
\pgfpathlineto{\pgfqpoint{2.130372in}{2.240519in}}%
\pgfpathlineto{\pgfqpoint{2.242459in}{2.283632in}}%
\pgfpathlineto{\pgfqpoint{2.354545in}{2.310369in}}%
\pgfpathlineto{\pgfqpoint{2.466632in}{2.339204in}}%
\pgfpathlineto{\pgfqpoint{2.578719in}{2.423511in}}%
\pgfpathlineto{\pgfqpoint{2.690806in}{2.470443in}}%
\pgfpathlineto{\pgfqpoint{2.802893in}{2.517835in}}%
\pgfpathlineto{\pgfqpoint{2.914979in}{2.587466in}}%
\pgfpathlineto{\pgfqpoint{3.027066in}{2.608468in}}%
\pgfpathlineto{\pgfqpoint{3.139153in}{2.632554in}}%
\pgfpathlineto{\pgfqpoint{3.251240in}{2.658438in}}%
\pgfpathlineto{\pgfqpoint{3.363326in}{2.727905in}}%
\pgfpathlineto{\pgfqpoint{3.475413in}{2.722807in}}%
\pgfpathlineto{\pgfqpoint{3.587500in}{2.705966in}}%
\pgfpathlineto{\pgfqpoint{3.699587in}{2.769312in}}%
\pgfpathlineto{\pgfqpoint{3.811674in}{2.777067in}}%
\pgfpathlineto{\pgfqpoint{3.923760in}{2.795322in}}%
\pgfpathlineto{\pgfqpoint{4.035847in}{2.845021in}}%
\pgfpathlineto{\pgfqpoint{4.147934in}{2.830722in}}%
\pgfpathlineto{\pgfqpoint{4.260021in}{2.821705in}}%
\pgfpathlineto{\pgfqpoint{4.372107in}{2.858022in}}%
\pgfpathlineto{\pgfqpoint{4.484194in}{2.900390in}}%
\pgfpathlineto{\pgfqpoint{4.596281in}{2.900669in}}%
\pgfpathlineto{\pgfqpoint{4.708368in}{2.903477in}}%
\pgfpathlineto{\pgfqpoint{4.820455in}{2.899729in}}%
\pgfpathlineto{\pgfqpoint{4.932541in}{2.889217in}}%
\pgfpathlineto{\pgfqpoint{5.044628in}{2.902871in}}%
\pgfpathlineto{\pgfqpoint{5.156715in}{2.921001in}}%
\pgfpathlineto{\pgfqpoint{5.268802in}{2.954180in}}%
\pgfpathlineto{\pgfqpoint{5.380888in}{2.941926in}}%
\pgfpathlineto{\pgfqpoint{5.492975in}{2.936950in}}%
\pgfpathlineto{\pgfqpoint{5.605062in}{2.926707in}}%
\pgfpathlineto{\pgfqpoint{5.717149in}{2.925351in}}%
\pgfpathlineto{\pgfqpoint{5.829236in}{2.935878in}}%
\pgfpathlineto{\pgfqpoint{5.941322in}{2.964802in}}%
\pgfpathlineto{\pgfqpoint{6.053409in}{2.977233in}}%
\pgfusepath{stroke}%
\end{pgfscope}%
\begin{pgfscope}%
\pgfpathrectangle{\pgfqpoint{0.875000in}{0.440000in}}{\pgfqpoint{5.425000in}{3.080000in}}%
\pgfusepath{clip}%
\pgfsetrectcap%
\pgfsetroundjoin%
\pgfsetlinewidth{1.505625pt}%
\definecolor{currentstroke}{rgb}{0.172549,0.627451,0.172549}%
\pgfsetstrokecolor{currentstroke}%
\pgfsetdash{}{0pt}%
\pgfpathmoveto{\pgfqpoint{1.121591in}{0.591695in}}%
\pgfpathlineto{\pgfqpoint{1.233678in}{0.882775in}}%
\pgfpathlineto{\pgfqpoint{1.345764in}{1.313313in}}%
\pgfpathlineto{\pgfqpoint{1.457851in}{1.606833in}}%
\pgfpathlineto{\pgfqpoint{1.569938in}{1.852613in}}%
\pgfpathlineto{\pgfqpoint{1.682025in}{2.044065in}}%
\pgfpathlineto{\pgfqpoint{1.794112in}{2.170504in}}%
\pgfpathlineto{\pgfqpoint{1.906198in}{2.344349in}}%
\pgfpathlineto{\pgfqpoint{2.018285in}{2.457863in}}%
\pgfpathlineto{\pgfqpoint{2.130372in}{2.628144in}}%
\pgfpathlineto{\pgfqpoint{2.242459in}{2.739498in}}%
\pgfpathlineto{\pgfqpoint{2.354545in}{2.762542in}}%
\pgfpathlineto{\pgfqpoint{2.466632in}{2.961023in}}%
\pgfpathlineto{\pgfqpoint{2.578719in}{2.916196in}}%
\pgfpathlineto{\pgfqpoint{2.690806in}{2.942286in}}%
\pgfpathlineto{\pgfqpoint{2.802893in}{3.076958in}}%
\pgfpathlineto{\pgfqpoint{2.914979in}{3.095885in}}%
\pgfpathlineto{\pgfqpoint{3.027066in}{3.155078in}}%
\pgfpathlineto{\pgfqpoint{3.139153in}{3.180644in}}%
\pgfpathlineto{\pgfqpoint{3.251240in}{3.162332in}}%
\pgfpathlineto{\pgfqpoint{3.363326in}{3.151018in}}%
\pgfpathlineto{\pgfqpoint{3.475413in}{3.159255in}}%
\pgfpathlineto{\pgfqpoint{3.587500in}{3.161949in}}%
\pgfpathlineto{\pgfqpoint{3.699587in}{3.120742in}}%
\pgfpathlineto{\pgfqpoint{3.811674in}{3.226537in}}%
\pgfpathlineto{\pgfqpoint{3.923760in}{3.160760in}}%
\pgfpathlineto{\pgfqpoint{4.035847in}{3.178010in}}%
\pgfpathlineto{\pgfqpoint{4.147934in}{3.163055in}}%
\pgfpathlineto{\pgfqpoint{4.260021in}{3.130229in}}%
\pgfpathlineto{\pgfqpoint{4.372107in}{3.187088in}}%
\pgfpathlineto{\pgfqpoint{4.484194in}{3.166750in}}%
\pgfpathlineto{\pgfqpoint{4.596281in}{3.252697in}}%
\pgfpathlineto{\pgfqpoint{4.708368in}{3.185257in}}%
\pgfpathlineto{\pgfqpoint{4.820455in}{3.178796in}}%
\pgfpathlineto{\pgfqpoint{4.932541in}{3.215919in}}%
\pgfpathlineto{\pgfqpoint{5.044628in}{3.130265in}}%
\pgfpathlineto{\pgfqpoint{5.156715in}{3.058678in}}%
\pgfpathlineto{\pgfqpoint{5.268802in}{3.175757in}}%
\pgfpathlineto{\pgfqpoint{5.380888in}{3.136306in}}%
\pgfpathlineto{\pgfqpoint{5.492975in}{3.294290in}}%
\pgfpathlineto{\pgfqpoint{5.605062in}{3.272709in}}%
\pgfpathlineto{\pgfqpoint{5.717149in}{3.202851in}}%
\pgfpathlineto{\pgfqpoint{5.829236in}{3.231690in}}%
\pgfpathlineto{\pgfqpoint{5.941322in}{3.224348in}}%
\pgfpathlineto{\pgfqpoint{6.053409in}{3.182074in}}%
\pgfusepath{stroke}%
\end{pgfscope}%
\begin{pgfscope}%
\pgfpathrectangle{\pgfqpoint{0.875000in}{0.440000in}}{\pgfqpoint{5.425000in}{3.080000in}}%
\pgfusepath{clip}%
\pgfsetrectcap%
\pgfsetroundjoin%
\pgfsetlinewidth{1.505625pt}%
\definecolor{currentstroke}{rgb}{0.839216,0.152941,0.156863}%
\pgfsetstrokecolor{currentstroke}%
\pgfsetdash{}{0pt}%
\pgfpathmoveto{\pgfqpoint{1.121591in}{0.580000in}}%
\pgfpathlineto{\pgfqpoint{1.233678in}{0.803507in}}%
\pgfpathlineto{\pgfqpoint{1.345764in}{1.127010in}}%
\pgfpathlineto{\pgfqpoint{1.457851in}{1.353141in}}%
\pgfpathlineto{\pgfqpoint{1.569938in}{1.616298in}}%
\pgfpathlineto{\pgfqpoint{1.682025in}{1.825786in}}%
\pgfpathlineto{\pgfqpoint{1.794112in}{1.995441in}}%
\pgfpathlineto{\pgfqpoint{1.906198in}{2.116216in}}%
\pgfpathlineto{\pgfqpoint{2.018285in}{2.208378in}}%
\pgfpathlineto{\pgfqpoint{2.130372in}{2.295069in}}%
\pgfpathlineto{\pgfqpoint{2.242459in}{2.416954in}}%
\pgfpathlineto{\pgfqpoint{2.354545in}{2.516432in}}%
\pgfpathlineto{\pgfqpoint{2.466632in}{2.596323in}}%
\pgfpathlineto{\pgfqpoint{2.578719in}{2.685392in}}%
\pgfpathlineto{\pgfqpoint{2.690806in}{2.693883in}}%
\pgfpathlineto{\pgfqpoint{2.802893in}{2.670101in}}%
\pgfpathlineto{\pgfqpoint{2.914979in}{2.761856in}}%
\pgfpathlineto{\pgfqpoint{3.027066in}{2.830363in}}%
\pgfpathlineto{\pgfqpoint{3.139153in}{2.915250in}}%
\pgfpathlineto{\pgfqpoint{3.251240in}{2.961761in}}%
\pgfpathlineto{\pgfqpoint{3.363326in}{2.951188in}}%
\pgfpathlineto{\pgfqpoint{3.475413in}{3.021331in}}%
\pgfpathlineto{\pgfqpoint{3.587500in}{2.950606in}}%
\pgfpathlineto{\pgfqpoint{3.699587in}{3.039663in}}%
\pgfpathlineto{\pgfqpoint{3.811674in}{3.044181in}}%
\pgfpathlineto{\pgfqpoint{3.923760in}{3.031809in}}%
\pgfpathlineto{\pgfqpoint{4.035847in}{3.050892in}}%
\pgfpathlineto{\pgfqpoint{4.147934in}{3.167862in}}%
\pgfpathlineto{\pgfqpoint{4.260021in}{3.152225in}}%
\pgfpathlineto{\pgfqpoint{4.372107in}{3.131782in}}%
\pgfpathlineto{\pgfqpoint{4.484194in}{3.125953in}}%
\pgfpathlineto{\pgfqpoint{4.596281in}{3.054681in}}%
\pgfpathlineto{\pgfqpoint{4.708368in}{3.148501in}}%
\pgfpathlineto{\pgfqpoint{4.820455in}{3.078978in}}%
\pgfpathlineto{\pgfqpoint{4.932541in}{3.145105in}}%
\pgfpathlineto{\pgfqpoint{5.044628in}{3.159238in}}%
\pgfpathlineto{\pgfqpoint{5.156715in}{3.133748in}}%
\pgfpathlineto{\pgfqpoint{5.268802in}{3.211372in}}%
\pgfpathlineto{\pgfqpoint{5.380888in}{3.100855in}}%
\pgfpathlineto{\pgfqpoint{5.492975in}{3.118200in}}%
\pgfpathlineto{\pgfqpoint{5.605062in}{3.083581in}}%
\pgfpathlineto{\pgfqpoint{5.717149in}{3.150881in}}%
\pgfpathlineto{\pgfqpoint{5.829236in}{3.143639in}}%
\pgfpathlineto{\pgfqpoint{5.941322in}{3.172443in}}%
\pgfpathlineto{\pgfqpoint{6.053409in}{3.168845in}}%
\pgfusepath{stroke}%
\end{pgfscope}%
\begin{pgfscope}%
\pgfpathrectangle{\pgfqpoint{0.875000in}{0.440000in}}{\pgfqpoint{5.425000in}{3.080000in}}%
\pgfusepath{clip}%
\pgfsetrectcap%
\pgfsetroundjoin%
\pgfsetlinewidth{1.505625pt}%
\definecolor{currentstroke}{rgb}{0.580392,0.403922,0.741176}%
\pgfsetstrokecolor{currentstroke}%
\pgfsetdash{}{0pt}%
\pgfpathmoveto{\pgfqpoint{1.121591in}{0.639769in}}%
\pgfpathlineto{\pgfqpoint{1.233678in}{1.093957in}}%
\pgfpathlineto{\pgfqpoint{1.345764in}{1.656795in}}%
\pgfpathlineto{\pgfqpoint{1.457851in}{2.101539in}}%
\pgfpathlineto{\pgfqpoint{1.569938in}{2.408292in}}%
\pgfpathlineto{\pgfqpoint{1.682025in}{2.573609in}}%
\pgfpathlineto{\pgfqpoint{1.794112in}{2.738487in}}%
\pgfpathlineto{\pgfqpoint{1.906198in}{2.822770in}}%
\pgfpathlineto{\pgfqpoint{2.018285in}{3.026883in}}%
\pgfpathlineto{\pgfqpoint{2.130372in}{3.220540in}}%
\pgfpathlineto{\pgfqpoint{2.242459in}{3.178296in}}%
\pgfpathlineto{\pgfqpoint{2.354545in}{3.195939in}}%
\pgfpathlineto{\pgfqpoint{2.466632in}{3.282921in}}%
\pgfpathlineto{\pgfqpoint{2.578719in}{3.375577in}}%
\pgfpathlineto{\pgfqpoint{2.690806in}{3.305021in}}%
\pgfpathlineto{\pgfqpoint{2.802893in}{3.340196in}}%
\pgfpathlineto{\pgfqpoint{2.914979in}{3.334386in}}%
\pgfpathlineto{\pgfqpoint{3.027066in}{3.279123in}}%
\pgfpathlineto{\pgfqpoint{3.139153in}{3.318571in}}%
\pgfpathlineto{\pgfqpoint{3.251240in}{3.325257in}}%
\pgfpathlineto{\pgfqpoint{3.363326in}{3.358281in}}%
\pgfpathlineto{\pgfqpoint{3.475413in}{3.334700in}}%
\pgfpathlineto{\pgfqpoint{3.587500in}{3.380000in}}%
\pgfpathlineto{\pgfqpoint{3.699587in}{3.305650in}}%
\pgfpathlineto{\pgfqpoint{3.811674in}{3.229410in}}%
\pgfpathlineto{\pgfqpoint{3.923760in}{3.273604in}}%
\pgfpathlineto{\pgfqpoint{4.035847in}{3.293667in}}%
\pgfpathlineto{\pgfqpoint{4.147934in}{3.279072in}}%
\pgfpathlineto{\pgfqpoint{4.260021in}{3.328606in}}%
\pgfpathlineto{\pgfqpoint{4.372107in}{3.321667in}}%
\pgfpathlineto{\pgfqpoint{4.484194in}{3.296235in}}%
\pgfpathlineto{\pgfqpoint{4.596281in}{3.354814in}}%
\pgfpathlineto{\pgfqpoint{4.708368in}{3.219488in}}%
\pgfpathlineto{\pgfqpoint{4.820455in}{3.244387in}}%
\pgfpathlineto{\pgfqpoint{4.932541in}{3.215811in}}%
\pgfpathlineto{\pgfqpoint{5.044628in}{3.360403in}}%
\pgfpathlineto{\pgfqpoint{5.156715in}{3.321280in}}%
\pgfpathlineto{\pgfqpoint{5.268802in}{3.283906in}}%
\pgfpathlineto{\pgfqpoint{5.380888in}{3.299952in}}%
\pgfpathlineto{\pgfqpoint{5.492975in}{3.166286in}}%
\pgfpathlineto{\pgfqpoint{5.605062in}{3.129849in}}%
\pgfpathlineto{\pgfqpoint{5.717149in}{3.152619in}}%
\pgfpathlineto{\pgfqpoint{5.829236in}{3.201982in}}%
\pgfpathlineto{\pgfqpoint{5.941322in}{3.224898in}}%
\pgfpathlineto{\pgfqpoint{6.053409in}{3.266513in}}%
\pgfusepath{stroke}%
\end{pgfscope}%
\begin{pgfscope}%
\pgfsetrectcap%
\pgfsetmiterjoin%
\pgfsetlinewidth{0.803000pt}%
\definecolor{currentstroke}{rgb}{0.000000,0.000000,0.000000}%
\pgfsetstrokecolor{currentstroke}%
\pgfsetdash{}{0pt}%
\pgfpathmoveto{\pgfqpoint{0.875000in}{0.440000in}}%
\pgfpathlineto{\pgfqpoint{0.875000in}{3.520000in}}%
\pgfusepath{stroke}%
\end{pgfscope}%
\begin{pgfscope}%
\pgfsetrectcap%
\pgfsetmiterjoin%
\pgfsetlinewidth{0.803000pt}%
\definecolor{currentstroke}{rgb}{0.000000,0.000000,0.000000}%
\pgfsetstrokecolor{currentstroke}%
\pgfsetdash{}{0pt}%
\pgfpathmoveto{\pgfqpoint{6.300000in}{0.440000in}}%
\pgfpathlineto{\pgfqpoint{6.300000in}{3.520000in}}%
\pgfusepath{stroke}%
\end{pgfscope}%
\begin{pgfscope}%
\pgfsetrectcap%
\pgfsetmiterjoin%
\pgfsetlinewidth{0.803000pt}%
\definecolor{currentstroke}{rgb}{0.000000,0.000000,0.000000}%
\pgfsetstrokecolor{currentstroke}%
\pgfsetdash{}{0pt}%
\pgfpathmoveto{\pgfqpoint{0.875000in}{0.440000in}}%
\pgfpathlineto{\pgfqpoint{6.300000in}{0.440000in}}%
\pgfusepath{stroke}%
\end{pgfscope}%
\begin{pgfscope}%
\pgfsetrectcap%
\pgfsetmiterjoin%
\pgfsetlinewidth{0.803000pt}%
\definecolor{currentstroke}{rgb}{0.000000,0.000000,0.000000}%
\pgfsetstrokecolor{currentstroke}%
\pgfsetdash{}{0pt}%
\pgfpathmoveto{\pgfqpoint{0.875000in}{3.520000in}}%
\pgfpathlineto{\pgfqpoint{6.300000in}{3.520000in}}%
\pgfusepath{stroke}%
\end{pgfscope}%
\begin{pgfscope}%
\pgfsetbuttcap%
\pgfsetmiterjoin%
\definecolor{currentfill}{rgb}{1.000000,1.000000,1.000000}%
\pgfsetfillcolor{currentfill}%
\pgfsetfillopacity{0.800000}%
\pgfsetlinewidth{1.003750pt}%
\definecolor{currentstroke}{rgb}{0.800000,0.800000,0.800000}%
\pgfsetstrokecolor{currentstroke}%
\pgfsetstrokeopacity{0.800000}%
\pgfsetdash{}{0pt}%
\pgfpathmoveto{\pgfqpoint{5.124999in}{0.509444in}}%
\pgfpathlineto{\pgfqpoint{6.202778in}{0.509444in}}%
\pgfpathquadraticcurveto{\pgfqpoint{6.230556in}{0.509444in}}{\pgfqpoint{6.230556in}{0.537222in}}%
\pgfpathlineto{\pgfqpoint{6.230556in}{1.491697in}}%
\pgfpathquadraticcurveto{\pgfqpoint{6.230556in}{1.519475in}}{\pgfqpoint{6.202778in}{1.519475in}}%
\pgfpathlineto{\pgfqpoint{5.124999in}{1.519475in}}%
\pgfpathquadraticcurveto{\pgfqpoint{5.097221in}{1.519475in}}{\pgfqpoint{5.097221in}{1.491697in}}%
\pgfpathlineto{\pgfqpoint{5.097221in}{0.537222in}}%
\pgfpathquadraticcurveto{\pgfqpoint{5.097221in}{0.509444in}}{\pgfqpoint{5.124999in}{0.509444in}}%
\pgfpathclose%
\pgfusepath{stroke,fill}%
\end{pgfscope}%
\begin{pgfscope}%
\pgfsetrectcap%
\pgfsetroundjoin%
\pgfsetlinewidth{1.505625pt}%
\definecolor{currentstroke}{rgb}{0.121569,0.466667,0.705882}%
\pgfsetstrokecolor{currentstroke}%
\pgfsetdash{}{0pt}%
\pgfpathmoveto{\pgfqpoint{5.152777in}{1.415308in}}%
\pgfpathlineto{\pgfqpoint{5.430554in}{1.415308in}}%
\pgfusepath{stroke}%
\end{pgfscope}%
\begin{pgfscope}%
\definecolor{textcolor}{rgb}{0.000000,0.000000,0.000000}%
\pgfsetstrokecolor{textcolor}%
\pgfsetfillcolor{textcolor}%
\pgftext[x=5.541666in,y=1.366697in,left,base]{\color{textcolor}\rmfamily\fontsize{10.000000}{12.000000}\selectfont VC\_Vision}%
\end{pgfscope}%
\begin{pgfscope}%
\pgfsetrectcap%
\pgfsetroundjoin%
\pgfsetlinewidth{1.505625pt}%
\definecolor{currentstroke}{rgb}{1.000000,0.498039,0.054902}%
\pgfsetstrokecolor{currentstroke}%
\pgfsetdash{}{0pt}%
\pgfpathmoveto{\pgfqpoint{5.152777in}{1.221636in}}%
\pgfpathlineto{\pgfqpoint{5.430554in}{1.221636in}}%
\pgfusepath{stroke}%
\end{pgfscope}%
\begin{pgfscope}%
\definecolor{textcolor}{rgb}{0.000000,0.000000,0.000000}%
\pgfsetstrokecolor{textcolor}%
\pgfsetfillcolor{textcolor}%
\pgftext[x=5.541666in,y=1.173024in,left,base]{\color{textcolor}\rmfamily\fontsize{10.000000}{12.000000}\selectfont VC\_Proj}%
\end{pgfscope}%
\begin{pgfscope}%
\pgfsetrectcap%
\pgfsetroundjoin%
\pgfsetlinewidth{1.505625pt}%
\definecolor{currentstroke}{rgb}{0.172549,0.627451,0.172549}%
\pgfsetstrokecolor{currentstroke}%
\pgfsetdash{}{0pt}%
\pgfpathmoveto{\pgfqpoint{5.152777in}{1.027963in}}%
\pgfpathlineto{\pgfqpoint{5.430554in}{1.027963in}}%
\pgfusepath{stroke}%
\end{pgfscope}%
\begin{pgfscope}%
\definecolor{textcolor}{rgb}{0.000000,0.000000,0.000000}%
\pgfsetstrokecolor{textcolor}%
\pgfsetfillcolor{textcolor}%
\pgftext[x=5.541666in,y=0.979352in,left,base]{\color{textcolor}\rmfamily\fontsize{10.000000}{12.000000}\selectfont IC}%
\end{pgfscope}%
\begin{pgfscope}%
\pgfsetrectcap%
\pgfsetroundjoin%
\pgfsetlinewidth{1.505625pt}%
\definecolor{currentstroke}{rgb}{0.839216,0.152941,0.156863}%
\pgfsetstrokecolor{currentstroke}%
\pgfsetdash{}{0pt}%
\pgfpathmoveto{\pgfqpoint{5.152777in}{0.834290in}}%
\pgfpathlineto{\pgfqpoint{5.430554in}{0.834290in}}%
\pgfusepath{stroke}%
\end{pgfscope}%
\begin{pgfscope}%
\definecolor{textcolor}{rgb}{0.000000,0.000000,0.000000}%
\pgfsetstrokecolor{textcolor}%
\pgfsetfillcolor{textcolor}%
\pgftext[x=5.541666in,y=0.785679in,left,base]{\color{textcolor}\rmfamily\fontsize{10.000000}{12.000000}\selectfont VC\_AT}%
\end{pgfscope}%
\begin{pgfscope}%
\pgfsetrectcap%
\pgfsetroundjoin%
\pgfsetlinewidth{1.505625pt}%
\definecolor{currentstroke}{rgb}{0.580392,0.403922,0.741176}%
\pgfsetstrokecolor{currentstroke}%
\pgfsetdash{}{0pt}%
\pgfpathmoveto{\pgfqpoint{5.152777in}{0.640617in}}%
\pgfpathlineto{\pgfqpoint{5.430554in}{0.640617in}}%
\pgfusepath{stroke}%
\end{pgfscope}%
\begin{pgfscope}%
\definecolor{textcolor}{rgb}{0.000000,0.000000,0.000000}%
\pgfsetstrokecolor{textcolor}%
\pgfsetfillcolor{textcolor}%
\pgftext[x=5.541666in,y=0.592006in,left,base]{\color{textcolor}\rmfamily\fontsize{10.000000}{12.000000}\selectfont VC\_DF}%
\end{pgfscope}%
\end{pgfpicture}%
\makeatother%
\endgroup%
}
    \caption[mAP of VC\_Vision, VC\_Proj, IC, VC\_AT, VC\_DF on each Epoch]{This chart illustrates the mAP of VC\_Vision, VC\_Proj, IC, VC\_AT, VC\_DF on each epoch.}
    \label{fig:ablation_vc}
\end{figure}

% \subsection{The effect smaller of batch size}
% Owing to the 80 Gb limitation of A100 GPU, I am able to train the fully learnable model, VC\_Vision, with a batch size of 16. To investigate the effect of this, I compare the 


% 0.514509499	0.543613255	0.559846222	0.257423162	0.481209993
% VC_AT	    IC	        VC_dd	    VC2_Vision	VC2_Proj
