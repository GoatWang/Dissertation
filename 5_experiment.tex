\section{CLIP with Prompt Engineering}


\subsection{Video CLIP or Image CLIP?}
Before entering the experiment on the long-tail issue, I first need to decide whether to use video CLIP or image CLIP for action recognition. To make this decision, I experiment with three model settings: 

\begin{enumerate}
    \item VC\_Vision: Video Clip trained on all vision layers, as illustrated in Figure \ref{fig:modelstructic1} with a learnable video encoder.
    \item VC\_Proj: Video Clip trained only on projection layers, as illustrated in Figure \ref{fig:modelstructic1} with a frozen video encoder.
    \item IC: Image Clip trained on post-transformer layers, as illustrated in Figure \ref{fig:modelstructic2} with a frozen video encoder.
\end{enumerate}

For all three of these experiments, models are trained with a binary cross-entropy loss function and the adamw optimizer, using a learning rate of 0.00015 on a single A100 GPU. They are all trained for 70 epochs. Due to memory limitations, I use a batch size of 16 for the VC\_Vision model, but 128 for the VC\_Proj and IC models. Referring to animal kingdom settings \parencite{ng2022animal}, the evaluation metrics used for this task include mean Average Precision (mAP) on the overall, head, middle, and tail classes, as provided by the dataset. 

% TODO
%The effect of batch size for Video CLIP will be further discussed in Section \ref{sec:ablation_bs}.

The results of the three experiments are shown in Table \ref{tab:resultsbackbone}, and the performance on each epoch during the training process is shown in Figure \ref{fig:tp_backbone}. As illustrated in the figure, the IC model significantly outperforms the other two models. Specifically, the VC\_vision model achieves a 27.19\% mAP on the overall dataset, the lowest among all the experiments. The VC\_Proj model attains a 49.86\% mAP on the overall dataset, making it the second lowest among all the experiments. Conversely, the IC model achieves a 54.79\% mAP on the overall dataset, the highest among all the other experiments. Although both the IC and VC\_Proj models already outperform the baseline model, CARe, which registers a 30.55\% mAP on the overall dataset, the IC model surpasses the VC\_Proj model by 4.81\% mAP on the overall dataset. Therefore, I have chosen to use the IC model for the following experiments. For further discussion about the performance of Image CLIP and Video CLIP, please refer to Section \ref{sec:ablation_vc}.

\begin{table}[ht]
    \centering
    \caption{Training Results for Visual Encoder Selection}
    \label{tab:resultsbackbone}
    \begin{tabular}{lllll}
        \toprule
        \multirow{2}{*}{Models} & \multicolumn{4}{c}{mAP} \\
        \cmidrule{2-5} 
        {} & Overall & Head  & Middle & Tail \\
        \midrule
        CARe        & 30.55   & 63.33 & 38.62 & 25.09 \\
        VC\_Vision  & 27.19   & 46.23 & 36.72 & 19.78 \\
        VC\_Proj    & 49.86   & 59.31 & 54.13 & 45.80 \\
        IC          & \textbf{54.79}   & \textbf{71.73} & \textbf{63.31} & \textbf{49.07} \\
        \bottomrule
    \end{tabular}
\end{table}

\begin{figure}[ht]
    \centering
    % \includegraphics[width=1.0\textwidth]{assets/imgs/5_1_BackboneSelection.pgf}
    \resizebox{1.0\textwidth}{!}{%% Creator: Matplotlib, PGF backend
%%
%% To include the figure in your LaTeX document, write
%%   \input{<filename>.pgf}
%%
%% Make sure the required packages are loaded in your preamble
%%   \usepackage{pgf}
%%
%% Also ensure that all the required font packages are loaded; for instance,
%% the lmodern package is sometimes necessary when using math font.
%%   \usepackage{lmodern}
%%
%% Figures using additional raster images can only be included by \input if
%% they are in the same directory as the main LaTeX file. For loading figures
%% from other directories you can use the `import` package
%%   \usepackage{import}
%%
%% and then include the figures with
%%   \import{<path to file>}{<filename>.pgf}
%%
%% Matplotlib used the following preamble
%%
\begingroup%
\makeatletter%
\begin{pgfpicture}%
\pgfpathrectangle{\pgfpointorigin}{\pgfqpoint{7.000000in}{4.000000in}}%
\pgfusepath{use as bounding box, clip}%
\begin{pgfscope}%
\pgfsetbuttcap%
\pgfsetmiterjoin%
\definecolor{currentfill}{rgb}{1.000000,1.000000,1.000000}%
\pgfsetfillcolor{currentfill}%
\pgfsetlinewidth{0.000000pt}%
\definecolor{currentstroke}{rgb}{1.000000,1.000000,1.000000}%
\pgfsetstrokecolor{currentstroke}%
\pgfsetdash{}{0pt}%
\pgfpathmoveto{\pgfqpoint{0.000000in}{0.000000in}}%
\pgfpathlineto{\pgfqpoint{7.000000in}{0.000000in}}%
\pgfpathlineto{\pgfqpoint{7.000000in}{4.000000in}}%
\pgfpathlineto{\pgfqpoint{0.000000in}{4.000000in}}%
\pgfpathlineto{\pgfqpoint{0.000000in}{0.000000in}}%
\pgfpathclose%
\pgfusepath{fill}%
\end{pgfscope}%
\begin{pgfscope}%
\pgfsetbuttcap%
\pgfsetmiterjoin%
\definecolor{currentfill}{rgb}{1.000000,1.000000,1.000000}%
\pgfsetfillcolor{currentfill}%
\pgfsetlinewidth{0.000000pt}%
\definecolor{currentstroke}{rgb}{0.000000,0.000000,0.000000}%
\pgfsetstrokecolor{currentstroke}%
\pgfsetstrokeopacity{0.000000}%
\pgfsetdash{}{0pt}%
\pgfpathmoveto{\pgfqpoint{0.875000in}{0.440000in}}%
\pgfpathlineto{\pgfqpoint{6.300000in}{0.440000in}}%
\pgfpathlineto{\pgfqpoint{6.300000in}{3.520000in}}%
\pgfpathlineto{\pgfqpoint{0.875000in}{3.520000in}}%
\pgfpathlineto{\pgfqpoint{0.875000in}{0.440000in}}%
\pgfpathclose%
\pgfusepath{fill}%
\end{pgfscope}%
\begin{pgfscope}%
\pgfpathrectangle{\pgfqpoint{0.875000in}{0.440000in}}{\pgfqpoint{5.425000in}{3.080000in}}%
\pgfusepath{clip}%
\pgfsetrectcap%
\pgfsetroundjoin%
\pgfsetlinewidth{0.803000pt}%
\definecolor{currentstroke}{rgb}{0.690196,0.690196,0.690196}%
\pgfsetstrokecolor{currentstroke}%
\pgfsetdash{}{0pt}%
\pgfpathmoveto{\pgfqpoint{1.121591in}{0.440000in}}%
\pgfpathlineto{\pgfqpoint{1.121591in}{3.520000in}}%
\pgfusepath{stroke}%
\end{pgfscope}%
\begin{pgfscope}%
\pgfsetbuttcap%
\pgfsetroundjoin%
\definecolor{currentfill}{rgb}{0.000000,0.000000,0.000000}%
\pgfsetfillcolor{currentfill}%
\pgfsetlinewidth{0.803000pt}%
\definecolor{currentstroke}{rgb}{0.000000,0.000000,0.000000}%
\pgfsetstrokecolor{currentstroke}%
\pgfsetdash{}{0pt}%
\pgfsys@defobject{currentmarker}{\pgfqpoint{0.000000in}{-0.048611in}}{\pgfqpoint{0.000000in}{0.000000in}}{%
\pgfpathmoveto{\pgfqpoint{0.000000in}{0.000000in}}%
\pgfpathlineto{\pgfqpoint{0.000000in}{-0.048611in}}%
\pgfusepath{stroke,fill}%
}%
\begin{pgfscope}%
\pgfsys@transformshift{1.121591in}{0.440000in}%
\pgfsys@useobject{currentmarker}{}%
\end{pgfscope}%
\end{pgfscope}%
\begin{pgfscope}%
\definecolor{textcolor}{rgb}{0.000000,0.000000,0.000000}%
\pgfsetstrokecolor{textcolor}%
\pgfsetfillcolor{textcolor}%
\pgftext[x=1.121591in,y=0.342778in,,top]{\color{textcolor}\rmfamily\fontsize{10.000000}{12.000000}\selectfont \(\displaystyle {0}\)}%
\end{pgfscope}%
\begin{pgfscope}%
\pgfpathrectangle{\pgfqpoint{0.875000in}{0.440000in}}{\pgfqpoint{5.425000in}{3.080000in}}%
\pgfusepath{clip}%
\pgfsetrectcap%
\pgfsetroundjoin%
\pgfsetlinewidth{0.803000pt}%
\definecolor{currentstroke}{rgb}{0.690196,0.690196,0.690196}%
\pgfsetstrokecolor{currentstroke}%
\pgfsetdash{}{0pt}%
\pgfpathmoveto{\pgfqpoint{1.826136in}{0.440000in}}%
\pgfpathlineto{\pgfqpoint{1.826136in}{3.520000in}}%
\pgfusepath{stroke}%
\end{pgfscope}%
\begin{pgfscope}%
\pgfsetbuttcap%
\pgfsetroundjoin%
\definecolor{currentfill}{rgb}{0.000000,0.000000,0.000000}%
\pgfsetfillcolor{currentfill}%
\pgfsetlinewidth{0.803000pt}%
\definecolor{currentstroke}{rgb}{0.000000,0.000000,0.000000}%
\pgfsetstrokecolor{currentstroke}%
\pgfsetdash{}{0pt}%
\pgfsys@defobject{currentmarker}{\pgfqpoint{0.000000in}{-0.048611in}}{\pgfqpoint{0.000000in}{0.000000in}}{%
\pgfpathmoveto{\pgfqpoint{0.000000in}{0.000000in}}%
\pgfpathlineto{\pgfqpoint{0.000000in}{-0.048611in}}%
\pgfusepath{stroke,fill}%
}%
\begin{pgfscope}%
\pgfsys@transformshift{1.826136in}{0.440000in}%
\pgfsys@useobject{currentmarker}{}%
\end{pgfscope}%
\end{pgfscope}%
\begin{pgfscope}%
\definecolor{textcolor}{rgb}{0.000000,0.000000,0.000000}%
\pgfsetstrokecolor{textcolor}%
\pgfsetfillcolor{textcolor}%
\pgftext[x=1.826136in,y=0.342778in,,top]{\color{textcolor}\rmfamily\fontsize{10.000000}{12.000000}\selectfont \(\displaystyle {10}\)}%
\end{pgfscope}%
\begin{pgfscope}%
\pgfpathrectangle{\pgfqpoint{0.875000in}{0.440000in}}{\pgfqpoint{5.425000in}{3.080000in}}%
\pgfusepath{clip}%
\pgfsetrectcap%
\pgfsetroundjoin%
\pgfsetlinewidth{0.803000pt}%
\definecolor{currentstroke}{rgb}{0.690196,0.690196,0.690196}%
\pgfsetstrokecolor{currentstroke}%
\pgfsetdash{}{0pt}%
\pgfpathmoveto{\pgfqpoint{2.530682in}{0.440000in}}%
\pgfpathlineto{\pgfqpoint{2.530682in}{3.520000in}}%
\pgfusepath{stroke}%
\end{pgfscope}%
\begin{pgfscope}%
\pgfsetbuttcap%
\pgfsetroundjoin%
\definecolor{currentfill}{rgb}{0.000000,0.000000,0.000000}%
\pgfsetfillcolor{currentfill}%
\pgfsetlinewidth{0.803000pt}%
\definecolor{currentstroke}{rgb}{0.000000,0.000000,0.000000}%
\pgfsetstrokecolor{currentstroke}%
\pgfsetdash{}{0pt}%
\pgfsys@defobject{currentmarker}{\pgfqpoint{0.000000in}{-0.048611in}}{\pgfqpoint{0.000000in}{0.000000in}}{%
\pgfpathmoveto{\pgfqpoint{0.000000in}{0.000000in}}%
\pgfpathlineto{\pgfqpoint{0.000000in}{-0.048611in}}%
\pgfusepath{stroke,fill}%
}%
\begin{pgfscope}%
\pgfsys@transformshift{2.530682in}{0.440000in}%
\pgfsys@useobject{currentmarker}{}%
\end{pgfscope}%
\end{pgfscope}%
\begin{pgfscope}%
\definecolor{textcolor}{rgb}{0.000000,0.000000,0.000000}%
\pgfsetstrokecolor{textcolor}%
\pgfsetfillcolor{textcolor}%
\pgftext[x=2.530682in,y=0.342778in,,top]{\color{textcolor}\rmfamily\fontsize{10.000000}{12.000000}\selectfont \(\displaystyle {20}\)}%
\end{pgfscope}%
\begin{pgfscope}%
\pgfpathrectangle{\pgfqpoint{0.875000in}{0.440000in}}{\pgfqpoint{5.425000in}{3.080000in}}%
\pgfusepath{clip}%
\pgfsetrectcap%
\pgfsetroundjoin%
\pgfsetlinewidth{0.803000pt}%
\definecolor{currentstroke}{rgb}{0.690196,0.690196,0.690196}%
\pgfsetstrokecolor{currentstroke}%
\pgfsetdash{}{0pt}%
\pgfpathmoveto{\pgfqpoint{3.235227in}{0.440000in}}%
\pgfpathlineto{\pgfqpoint{3.235227in}{3.520000in}}%
\pgfusepath{stroke}%
\end{pgfscope}%
\begin{pgfscope}%
\pgfsetbuttcap%
\pgfsetroundjoin%
\definecolor{currentfill}{rgb}{0.000000,0.000000,0.000000}%
\pgfsetfillcolor{currentfill}%
\pgfsetlinewidth{0.803000pt}%
\definecolor{currentstroke}{rgb}{0.000000,0.000000,0.000000}%
\pgfsetstrokecolor{currentstroke}%
\pgfsetdash{}{0pt}%
\pgfsys@defobject{currentmarker}{\pgfqpoint{0.000000in}{-0.048611in}}{\pgfqpoint{0.000000in}{0.000000in}}{%
\pgfpathmoveto{\pgfqpoint{0.000000in}{0.000000in}}%
\pgfpathlineto{\pgfqpoint{0.000000in}{-0.048611in}}%
\pgfusepath{stroke,fill}%
}%
\begin{pgfscope}%
\pgfsys@transformshift{3.235227in}{0.440000in}%
\pgfsys@useobject{currentmarker}{}%
\end{pgfscope}%
\end{pgfscope}%
\begin{pgfscope}%
\definecolor{textcolor}{rgb}{0.000000,0.000000,0.000000}%
\pgfsetstrokecolor{textcolor}%
\pgfsetfillcolor{textcolor}%
\pgftext[x=3.235227in,y=0.342778in,,top]{\color{textcolor}\rmfamily\fontsize{10.000000}{12.000000}\selectfont \(\displaystyle {30}\)}%
\end{pgfscope}%
\begin{pgfscope}%
\pgfpathrectangle{\pgfqpoint{0.875000in}{0.440000in}}{\pgfqpoint{5.425000in}{3.080000in}}%
\pgfusepath{clip}%
\pgfsetrectcap%
\pgfsetroundjoin%
\pgfsetlinewidth{0.803000pt}%
\definecolor{currentstroke}{rgb}{0.690196,0.690196,0.690196}%
\pgfsetstrokecolor{currentstroke}%
\pgfsetdash{}{0pt}%
\pgfpathmoveto{\pgfqpoint{3.939773in}{0.440000in}}%
\pgfpathlineto{\pgfqpoint{3.939773in}{3.520000in}}%
\pgfusepath{stroke}%
\end{pgfscope}%
\begin{pgfscope}%
\pgfsetbuttcap%
\pgfsetroundjoin%
\definecolor{currentfill}{rgb}{0.000000,0.000000,0.000000}%
\pgfsetfillcolor{currentfill}%
\pgfsetlinewidth{0.803000pt}%
\definecolor{currentstroke}{rgb}{0.000000,0.000000,0.000000}%
\pgfsetstrokecolor{currentstroke}%
\pgfsetdash{}{0pt}%
\pgfsys@defobject{currentmarker}{\pgfqpoint{0.000000in}{-0.048611in}}{\pgfqpoint{0.000000in}{0.000000in}}{%
\pgfpathmoveto{\pgfqpoint{0.000000in}{0.000000in}}%
\pgfpathlineto{\pgfqpoint{0.000000in}{-0.048611in}}%
\pgfusepath{stroke,fill}%
}%
\begin{pgfscope}%
\pgfsys@transformshift{3.939773in}{0.440000in}%
\pgfsys@useobject{currentmarker}{}%
\end{pgfscope}%
\end{pgfscope}%
\begin{pgfscope}%
\definecolor{textcolor}{rgb}{0.000000,0.000000,0.000000}%
\pgfsetstrokecolor{textcolor}%
\pgfsetfillcolor{textcolor}%
\pgftext[x=3.939773in,y=0.342778in,,top]{\color{textcolor}\rmfamily\fontsize{10.000000}{12.000000}\selectfont \(\displaystyle {40}\)}%
\end{pgfscope}%
\begin{pgfscope}%
\pgfpathrectangle{\pgfqpoint{0.875000in}{0.440000in}}{\pgfqpoint{5.425000in}{3.080000in}}%
\pgfusepath{clip}%
\pgfsetrectcap%
\pgfsetroundjoin%
\pgfsetlinewidth{0.803000pt}%
\definecolor{currentstroke}{rgb}{0.690196,0.690196,0.690196}%
\pgfsetstrokecolor{currentstroke}%
\pgfsetdash{}{0pt}%
\pgfpathmoveto{\pgfqpoint{4.644318in}{0.440000in}}%
\pgfpathlineto{\pgfqpoint{4.644318in}{3.520000in}}%
\pgfusepath{stroke}%
\end{pgfscope}%
\begin{pgfscope}%
\pgfsetbuttcap%
\pgfsetroundjoin%
\definecolor{currentfill}{rgb}{0.000000,0.000000,0.000000}%
\pgfsetfillcolor{currentfill}%
\pgfsetlinewidth{0.803000pt}%
\definecolor{currentstroke}{rgb}{0.000000,0.000000,0.000000}%
\pgfsetstrokecolor{currentstroke}%
\pgfsetdash{}{0pt}%
\pgfsys@defobject{currentmarker}{\pgfqpoint{0.000000in}{-0.048611in}}{\pgfqpoint{0.000000in}{0.000000in}}{%
\pgfpathmoveto{\pgfqpoint{0.000000in}{0.000000in}}%
\pgfpathlineto{\pgfqpoint{0.000000in}{-0.048611in}}%
\pgfusepath{stroke,fill}%
}%
\begin{pgfscope}%
\pgfsys@transformshift{4.644318in}{0.440000in}%
\pgfsys@useobject{currentmarker}{}%
\end{pgfscope}%
\end{pgfscope}%
\begin{pgfscope}%
\definecolor{textcolor}{rgb}{0.000000,0.000000,0.000000}%
\pgfsetstrokecolor{textcolor}%
\pgfsetfillcolor{textcolor}%
\pgftext[x=4.644318in,y=0.342778in,,top]{\color{textcolor}\rmfamily\fontsize{10.000000}{12.000000}\selectfont \(\displaystyle {50}\)}%
\end{pgfscope}%
\begin{pgfscope}%
\pgfpathrectangle{\pgfqpoint{0.875000in}{0.440000in}}{\pgfqpoint{5.425000in}{3.080000in}}%
\pgfusepath{clip}%
\pgfsetrectcap%
\pgfsetroundjoin%
\pgfsetlinewidth{0.803000pt}%
\definecolor{currentstroke}{rgb}{0.690196,0.690196,0.690196}%
\pgfsetstrokecolor{currentstroke}%
\pgfsetdash{}{0pt}%
\pgfpathmoveto{\pgfqpoint{5.348864in}{0.440000in}}%
\pgfpathlineto{\pgfqpoint{5.348864in}{3.520000in}}%
\pgfusepath{stroke}%
\end{pgfscope}%
\begin{pgfscope}%
\pgfsetbuttcap%
\pgfsetroundjoin%
\definecolor{currentfill}{rgb}{0.000000,0.000000,0.000000}%
\pgfsetfillcolor{currentfill}%
\pgfsetlinewidth{0.803000pt}%
\definecolor{currentstroke}{rgb}{0.000000,0.000000,0.000000}%
\pgfsetstrokecolor{currentstroke}%
\pgfsetdash{}{0pt}%
\pgfsys@defobject{currentmarker}{\pgfqpoint{0.000000in}{-0.048611in}}{\pgfqpoint{0.000000in}{0.000000in}}{%
\pgfpathmoveto{\pgfqpoint{0.000000in}{0.000000in}}%
\pgfpathlineto{\pgfqpoint{0.000000in}{-0.048611in}}%
\pgfusepath{stroke,fill}%
}%
\begin{pgfscope}%
\pgfsys@transformshift{5.348864in}{0.440000in}%
\pgfsys@useobject{currentmarker}{}%
\end{pgfscope}%
\end{pgfscope}%
\begin{pgfscope}%
\definecolor{textcolor}{rgb}{0.000000,0.000000,0.000000}%
\pgfsetstrokecolor{textcolor}%
\pgfsetfillcolor{textcolor}%
\pgftext[x=5.348864in,y=0.342778in,,top]{\color{textcolor}\rmfamily\fontsize{10.000000}{12.000000}\selectfont \(\displaystyle {60}\)}%
\end{pgfscope}%
\begin{pgfscope}%
\pgfpathrectangle{\pgfqpoint{0.875000in}{0.440000in}}{\pgfqpoint{5.425000in}{3.080000in}}%
\pgfusepath{clip}%
\pgfsetrectcap%
\pgfsetroundjoin%
\pgfsetlinewidth{0.803000pt}%
\definecolor{currentstroke}{rgb}{0.690196,0.690196,0.690196}%
\pgfsetstrokecolor{currentstroke}%
\pgfsetdash{}{0pt}%
\pgfpathmoveto{\pgfqpoint{6.053409in}{0.440000in}}%
\pgfpathlineto{\pgfqpoint{6.053409in}{3.520000in}}%
\pgfusepath{stroke}%
\end{pgfscope}%
\begin{pgfscope}%
\pgfsetbuttcap%
\pgfsetroundjoin%
\definecolor{currentfill}{rgb}{0.000000,0.000000,0.000000}%
\pgfsetfillcolor{currentfill}%
\pgfsetlinewidth{0.803000pt}%
\definecolor{currentstroke}{rgb}{0.000000,0.000000,0.000000}%
\pgfsetstrokecolor{currentstroke}%
\pgfsetdash{}{0pt}%
\pgfsys@defobject{currentmarker}{\pgfqpoint{0.000000in}{-0.048611in}}{\pgfqpoint{0.000000in}{0.000000in}}{%
\pgfpathmoveto{\pgfqpoint{0.000000in}{0.000000in}}%
\pgfpathlineto{\pgfqpoint{0.000000in}{-0.048611in}}%
\pgfusepath{stroke,fill}%
}%
\begin{pgfscope}%
\pgfsys@transformshift{6.053409in}{0.440000in}%
\pgfsys@useobject{currentmarker}{}%
\end{pgfscope}%
\end{pgfscope}%
\begin{pgfscope}%
\definecolor{textcolor}{rgb}{0.000000,0.000000,0.000000}%
\pgfsetstrokecolor{textcolor}%
\pgfsetfillcolor{textcolor}%
\pgftext[x=6.053409in,y=0.342778in,,top]{\color{textcolor}\rmfamily\fontsize{10.000000}{12.000000}\selectfont \(\displaystyle {70}\)}%
\end{pgfscope}%
\begin{pgfscope}%
\definecolor{textcolor}{rgb}{0.000000,0.000000,0.000000}%
\pgfsetstrokecolor{textcolor}%
\pgfsetfillcolor{textcolor}%
\pgftext[x=3.587500in,y=0.163766in,,top]{\color{textcolor}\rmfamily\fontsize{10.000000}{12.000000}\selectfont Epoch}%
\end{pgfscope}%
\begin{pgfscope}%
\pgfpathrectangle{\pgfqpoint{0.875000in}{0.440000in}}{\pgfqpoint{5.425000in}{3.080000in}}%
\pgfusepath{clip}%
\pgfsetrectcap%
\pgfsetroundjoin%
\pgfsetlinewidth{0.803000pt}%
\definecolor{currentstroke}{rgb}{0.690196,0.690196,0.690196}%
\pgfsetstrokecolor{currentstroke}%
\pgfsetdash{}{0pt}%
\pgfpathmoveto{\pgfqpoint{0.875000in}{0.935141in}}%
\pgfpathlineto{\pgfqpoint{6.300000in}{0.935141in}}%
\pgfusepath{stroke}%
\end{pgfscope}%
\begin{pgfscope}%
\pgfsetbuttcap%
\pgfsetroundjoin%
\definecolor{currentfill}{rgb}{0.000000,0.000000,0.000000}%
\pgfsetfillcolor{currentfill}%
\pgfsetlinewidth{0.803000pt}%
\definecolor{currentstroke}{rgb}{0.000000,0.000000,0.000000}%
\pgfsetstrokecolor{currentstroke}%
\pgfsetdash{}{0pt}%
\pgfsys@defobject{currentmarker}{\pgfqpoint{-0.048611in}{0.000000in}}{\pgfqpoint{-0.000000in}{0.000000in}}{%
\pgfpathmoveto{\pgfqpoint{-0.000000in}{0.000000in}}%
\pgfpathlineto{\pgfqpoint{-0.048611in}{0.000000in}}%
\pgfusepath{stroke,fill}%
}%
\begin{pgfscope}%
\pgfsys@transformshift{0.875000in}{0.935141in}%
\pgfsys@useobject{currentmarker}{}%
\end{pgfscope}%
\end{pgfscope}%
\begin{pgfscope}%
\definecolor{textcolor}{rgb}{0.000000,0.000000,0.000000}%
\pgfsetstrokecolor{textcolor}%
\pgfsetfillcolor{textcolor}%
\pgftext[x=0.600308in, y=0.886916in, left, base]{\color{textcolor}\rmfamily\fontsize{10.000000}{12.000000}\selectfont \(\displaystyle {0.1}\)}%
\end{pgfscope}%
\begin{pgfscope}%
\pgfpathrectangle{\pgfqpoint{0.875000in}{0.440000in}}{\pgfqpoint{5.425000in}{3.080000in}}%
\pgfusepath{clip}%
\pgfsetrectcap%
\pgfsetroundjoin%
\pgfsetlinewidth{0.803000pt}%
\definecolor{currentstroke}{rgb}{0.690196,0.690196,0.690196}%
\pgfsetstrokecolor{currentstroke}%
\pgfsetdash{}{0pt}%
\pgfpathmoveto{\pgfqpoint{0.875000in}{1.480963in}}%
\pgfpathlineto{\pgfqpoint{6.300000in}{1.480963in}}%
\pgfusepath{stroke}%
\end{pgfscope}%
\begin{pgfscope}%
\pgfsetbuttcap%
\pgfsetroundjoin%
\definecolor{currentfill}{rgb}{0.000000,0.000000,0.000000}%
\pgfsetfillcolor{currentfill}%
\pgfsetlinewidth{0.803000pt}%
\definecolor{currentstroke}{rgb}{0.000000,0.000000,0.000000}%
\pgfsetstrokecolor{currentstroke}%
\pgfsetdash{}{0pt}%
\pgfsys@defobject{currentmarker}{\pgfqpoint{-0.048611in}{0.000000in}}{\pgfqpoint{-0.000000in}{0.000000in}}{%
\pgfpathmoveto{\pgfqpoint{-0.000000in}{0.000000in}}%
\pgfpathlineto{\pgfqpoint{-0.048611in}{0.000000in}}%
\pgfusepath{stroke,fill}%
}%
\begin{pgfscope}%
\pgfsys@transformshift{0.875000in}{1.480963in}%
\pgfsys@useobject{currentmarker}{}%
\end{pgfscope}%
\end{pgfscope}%
\begin{pgfscope}%
\definecolor{textcolor}{rgb}{0.000000,0.000000,0.000000}%
\pgfsetstrokecolor{textcolor}%
\pgfsetfillcolor{textcolor}%
\pgftext[x=0.600308in, y=1.432738in, left, base]{\color{textcolor}\rmfamily\fontsize{10.000000}{12.000000}\selectfont \(\displaystyle {0.2}\)}%
\end{pgfscope}%
\begin{pgfscope}%
\pgfpathrectangle{\pgfqpoint{0.875000in}{0.440000in}}{\pgfqpoint{5.425000in}{3.080000in}}%
\pgfusepath{clip}%
\pgfsetrectcap%
\pgfsetroundjoin%
\pgfsetlinewidth{0.803000pt}%
\definecolor{currentstroke}{rgb}{0.690196,0.690196,0.690196}%
\pgfsetstrokecolor{currentstroke}%
\pgfsetdash{}{0pt}%
\pgfpathmoveto{\pgfqpoint{0.875000in}{2.026784in}}%
\pgfpathlineto{\pgfqpoint{6.300000in}{2.026784in}}%
\pgfusepath{stroke}%
\end{pgfscope}%
\begin{pgfscope}%
\pgfsetbuttcap%
\pgfsetroundjoin%
\definecolor{currentfill}{rgb}{0.000000,0.000000,0.000000}%
\pgfsetfillcolor{currentfill}%
\pgfsetlinewidth{0.803000pt}%
\definecolor{currentstroke}{rgb}{0.000000,0.000000,0.000000}%
\pgfsetstrokecolor{currentstroke}%
\pgfsetdash{}{0pt}%
\pgfsys@defobject{currentmarker}{\pgfqpoint{-0.048611in}{0.000000in}}{\pgfqpoint{-0.000000in}{0.000000in}}{%
\pgfpathmoveto{\pgfqpoint{-0.000000in}{0.000000in}}%
\pgfpathlineto{\pgfqpoint{-0.048611in}{0.000000in}}%
\pgfusepath{stroke,fill}%
}%
\begin{pgfscope}%
\pgfsys@transformshift{0.875000in}{2.026784in}%
\pgfsys@useobject{currentmarker}{}%
\end{pgfscope}%
\end{pgfscope}%
\begin{pgfscope}%
\definecolor{textcolor}{rgb}{0.000000,0.000000,0.000000}%
\pgfsetstrokecolor{textcolor}%
\pgfsetfillcolor{textcolor}%
\pgftext[x=0.600308in, y=1.978559in, left, base]{\color{textcolor}\rmfamily\fontsize{10.000000}{12.000000}\selectfont \(\displaystyle {0.3}\)}%
\end{pgfscope}%
\begin{pgfscope}%
\pgfpathrectangle{\pgfqpoint{0.875000in}{0.440000in}}{\pgfqpoint{5.425000in}{3.080000in}}%
\pgfusepath{clip}%
\pgfsetrectcap%
\pgfsetroundjoin%
\pgfsetlinewidth{0.803000pt}%
\definecolor{currentstroke}{rgb}{0.690196,0.690196,0.690196}%
\pgfsetstrokecolor{currentstroke}%
\pgfsetdash{}{0pt}%
\pgfpathmoveto{\pgfqpoint{0.875000in}{2.572606in}}%
\pgfpathlineto{\pgfqpoint{6.300000in}{2.572606in}}%
\pgfusepath{stroke}%
\end{pgfscope}%
\begin{pgfscope}%
\pgfsetbuttcap%
\pgfsetroundjoin%
\definecolor{currentfill}{rgb}{0.000000,0.000000,0.000000}%
\pgfsetfillcolor{currentfill}%
\pgfsetlinewidth{0.803000pt}%
\definecolor{currentstroke}{rgb}{0.000000,0.000000,0.000000}%
\pgfsetstrokecolor{currentstroke}%
\pgfsetdash{}{0pt}%
\pgfsys@defobject{currentmarker}{\pgfqpoint{-0.048611in}{0.000000in}}{\pgfqpoint{-0.000000in}{0.000000in}}{%
\pgfpathmoveto{\pgfqpoint{-0.000000in}{0.000000in}}%
\pgfpathlineto{\pgfqpoint{-0.048611in}{0.000000in}}%
\pgfusepath{stroke,fill}%
}%
\begin{pgfscope}%
\pgfsys@transformshift{0.875000in}{2.572606in}%
\pgfsys@useobject{currentmarker}{}%
\end{pgfscope}%
\end{pgfscope}%
\begin{pgfscope}%
\definecolor{textcolor}{rgb}{0.000000,0.000000,0.000000}%
\pgfsetstrokecolor{textcolor}%
\pgfsetfillcolor{textcolor}%
\pgftext[x=0.600308in, y=2.524380in, left, base]{\color{textcolor}\rmfamily\fontsize{10.000000}{12.000000}\selectfont \(\displaystyle {0.4}\)}%
\end{pgfscope}%
\begin{pgfscope}%
\pgfpathrectangle{\pgfqpoint{0.875000in}{0.440000in}}{\pgfqpoint{5.425000in}{3.080000in}}%
\pgfusepath{clip}%
\pgfsetrectcap%
\pgfsetroundjoin%
\pgfsetlinewidth{0.803000pt}%
\definecolor{currentstroke}{rgb}{0.690196,0.690196,0.690196}%
\pgfsetstrokecolor{currentstroke}%
\pgfsetdash{}{0pt}%
\pgfpathmoveto{\pgfqpoint{0.875000in}{3.118427in}}%
\pgfpathlineto{\pgfqpoint{6.300000in}{3.118427in}}%
\pgfusepath{stroke}%
\end{pgfscope}%
\begin{pgfscope}%
\pgfsetbuttcap%
\pgfsetroundjoin%
\definecolor{currentfill}{rgb}{0.000000,0.000000,0.000000}%
\pgfsetfillcolor{currentfill}%
\pgfsetlinewidth{0.803000pt}%
\definecolor{currentstroke}{rgb}{0.000000,0.000000,0.000000}%
\pgfsetstrokecolor{currentstroke}%
\pgfsetdash{}{0pt}%
\pgfsys@defobject{currentmarker}{\pgfqpoint{-0.048611in}{0.000000in}}{\pgfqpoint{-0.000000in}{0.000000in}}{%
\pgfpathmoveto{\pgfqpoint{-0.000000in}{0.000000in}}%
\pgfpathlineto{\pgfqpoint{-0.048611in}{0.000000in}}%
\pgfusepath{stroke,fill}%
}%
\begin{pgfscope}%
\pgfsys@transformshift{0.875000in}{3.118427in}%
\pgfsys@useobject{currentmarker}{}%
\end{pgfscope}%
\end{pgfscope}%
\begin{pgfscope}%
\definecolor{textcolor}{rgb}{0.000000,0.000000,0.000000}%
\pgfsetstrokecolor{textcolor}%
\pgfsetfillcolor{textcolor}%
\pgftext[x=0.600308in, y=3.070202in, left, base]{\color{textcolor}\rmfamily\fontsize{10.000000}{12.000000}\selectfont \(\displaystyle {0.5}\)}%
\end{pgfscope}%
\begin{pgfscope}%
\definecolor{textcolor}{rgb}{0.000000,0.000000,0.000000}%
\pgfsetstrokecolor{textcolor}%
\pgfsetfillcolor{textcolor}%
\pgftext[x=0.544752in,y=1.980000in,,bottom,rotate=90.000000]{\color{textcolor}\rmfamily\fontsize{10.000000}{12.000000}\selectfont mAP}%
\end{pgfscope}%
\begin{pgfscope}%
\pgfpathrectangle{\pgfqpoint{0.875000in}{0.440000in}}{\pgfqpoint{5.425000in}{3.080000in}}%
\pgfusepath{clip}%
\pgfsetrectcap%
\pgfsetroundjoin%
\pgfsetlinewidth{1.505625pt}%
\definecolor{currentstroke}{rgb}{0.121569,0.466667,0.705882}%
\pgfsetstrokecolor{currentstroke}%
\pgfsetdash{}{0pt}%
\pgfpathmoveto{\pgfqpoint{1.121591in}{0.772223in}}%
\pgfpathlineto{\pgfqpoint{1.192045in}{1.013329in}}%
\pgfpathlineto{\pgfqpoint{1.262500in}{1.079139in}}%
\pgfpathlineto{\pgfqpoint{1.332955in}{1.187777in}}%
\pgfpathlineto{\pgfqpoint{1.403409in}{1.231464in}}%
\pgfpathlineto{\pgfqpoint{1.473864in}{1.304834in}}%
\pgfpathlineto{\pgfqpoint{1.544318in}{1.321441in}}%
\pgfpathlineto{\pgfqpoint{1.614773in}{1.368052in}}%
\pgfpathlineto{\pgfqpoint{1.685227in}{1.406592in}}%
\pgfpathlineto{\pgfqpoint{1.755682in}{1.434296in}}%
\pgfpathlineto{\pgfqpoint{1.826136in}{1.404112in}}%
\pgfpathlineto{\pgfqpoint{1.896591in}{1.457361in}}%
\pgfpathlineto{\pgfqpoint{1.967045in}{1.469799in}}%
\pgfpathlineto{\pgfqpoint{2.037500in}{1.452448in}}%
\pgfpathlineto{\pgfqpoint{2.107955in}{1.505787in}}%
\pgfpathlineto{\pgfqpoint{2.178409in}{1.496825in}}%
\pgfpathlineto{\pgfqpoint{2.248864in}{1.599660in}}%
\pgfpathlineto{\pgfqpoint{2.319318in}{1.571746in}}%
\pgfpathlineto{\pgfqpoint{2.389773in}{1.597040in}}%
\pgfpathlineto{\pgfqpoint{2.460227in}{1.624680in}}%
\pgfpathlineto{\pgfqpoint{2.530682in}{1.583878in}}%
\pgfpathlineto{\pgfqpoint{2.601136in}{1.624191in}}%
\pgfpathlineto{\pgfqpoint{2.671591in}{1.627656in}}%
\pgfpathlineto{\pgfqpoint{2.742045in}{1.676802in}}%
\pgfpathlineto{\pgfqpoint{2.812500in}{1.685820in}}%
\pgfpathlineto{\pgfqpoint{2.882955in}{1.631826in}}%
\pgfpathlineto{\pgfqpoint{2.953409in}{1.633845in}}%
\pgfpathlineto{\pgfqpoint{3.023864in}{1.667289in}}%
\pgfpathlineto{\pgfqpoint{3.094318in}{1.709995in}}%
\pgfpathlineto{\pgfqpoint{3.164773in}{1.617126in}}%
\pgfpathlineto{\pgfqpoint{3.235227in}{1.626341in}}%
\pgfpathlineto{\pgfqpoint{3.305682in}{1.711931in}}%
\pgfpathlineto{\pgfqpoint{3.376136in}{1.758429in}}%
\pgfpathlineto{\pgfqpoint{3.446591in}{1.660277in}}%
\pgfpathlineto{\pgfqpoint{3.517045in}{1.666614in}}%
\pgfpathlineto{\pgfqpoint{3.587500in}{1.740843in}}%
\pgfpathlineto{\pgfqpoint{3.657955in}{1.768443in}}%
\pgfpathlineto{\pgfqpoint{3.728409in}{1.682276in}}%
\pgfpathlineto{\pgfqpoint{3.798864in}{1.683024in}}%
\pgfpathlineto{\pgfqpoint{3.869318in}{1.750034in}}%
\pgfpathlineto{\pgfqpoint{3.939773in}{1.784434in}}%
\pgfpathlineto{\pgfqpoint{4.010227in}{1.714556in}}%
\pgfpathlineto{\pgfqpoint{4.080682in}{1.789950in}}%
\pgfpathlineto{\pgfqpoint{4.151136in}{1.794391in}}%
\pgfpathlineto{\pgfqpoint{4.221591in}{1.731895in}}%
\pgfpathlineto{\pgfqpoint{4.292045in}{1.767166in}}%
\pgfpathlineto{\pgfqpoint{4.362500in}{1.812205in}}%
\pgfpathlineto{\pgfqpoint{4.432955in}{1.730770in}}%
\pgfpathlineto{\pgfqpoint{4.503409in}{1.780466in}}%
\pgfpathlineto{\pgfqpoint{4.573864in}{1.815041in}}%
\pgfpathlineto{\pgfqpoint{4.644318in}{1.762829in}}%
\pgfpathlineto{\pgfqpoint{4.714773in}{1.778884in}}%
\pgfpathlineto{\pgfqpoint{4.785227in}{1.812365in}}%
\pgfpathlineto{\pgfqpoint{4.855682in}{1.727521in}}%
\pgfpathlineto{\pgfqpoint{4.926136in}{1.743880in}}%
\pgfpathlineto{\pgfqpoint{4.996591in}{1.792384in}}%
\pgfpathlineto{\pgfqpoint{5.067045in}{1.772418in}}%
\pgfpathlineto{\pgfqpoint{5.137500in}{1.791562in}}%
\pgfpathlineto{\pgfqpoint{5.207955in}{1.828114in}}%
\pgfpathlineto{\pgfqpoint{5.278409in}{1.738947in}}%
\pgfpathlineto{\pgfqpoint{5.348864in}{1.764098in}}%
\pgfpathlineto{\pgfqpoint{5.419318in}{1.810368in}}%
\pgfpathlineto{\pgfqpoint{5.489773in}{1.757716in}}%
\pgfpathlineto{\pgfqpoint{5.560227in}{1.808434in}}%
\pgfpathlineto{\pgfqpoint{5.630682in}{1.825571in}}%
\pgfpathlineto{\pgfqpoint{5.701136in}{1.766845in}}%
\pgfpathlineto{\pgfqpoint{5.771591in}{1.777801in}}%
\pgfpathlineto{\pgfqpoint{5.842045in}{1.811782in}}%
\pgfpathlineto{\pgfqpoint{5.912500in}{1.858545in}}%
\pgfpathlineto{\pgfqpoint{5.982955in}{1.824746in}}%
\pgfpathlineto{\pgfqpoint{6.053409in}{1.830342in}}%
\pgfusepath{stroke}%
\end{pgfscope}%
\begin{pgfscope}%
\pgfpathrectangle{\pgfqpoint{0.875000in}{0.440000in}}{\pgfqpoint{5.425000in}{3.080000in}}%
\pgfusepath{clip}%
\pgfsetrectcap%
\pgfsetroundjoin%
\pgfsetlinewidth{1.505625pt}%
\definecolor{currentstroke}{rgb}{1.000000,0.498039,0.054902}%
\pgfsetstrokecolor{currentstroke}%
\pgfsetdash{}{0pt}%
\pgfpathmoveto{\pgfqpoint{1.121591in}{0.697206in}}%
\pgfpathlineto{\pgfqpoint{1.192045in}{0.973911in}}%
\pgfpathlineto{\pgfqpoint{1.262500in}{1.311937in}}%
\pgfpathlineto{\pgfqpoint{1.332955in}{1.570234in}}%
\pgfpathlineto{\pgfqpoint{1.403409in}{1.777719in}}%
\pgfpathlineto{\pgfqpoint{1.473864in}{1.938353in}}%
\pgfpathlineto{\pgfqpoint{1.544318in}{2.080406in}}%
\pgfpathlineto{\pgfqpoint{1.614773in}{2.183892in}}%
\pgfpathlineto{\pgfqpoint{1.685227in}{2.183273in}}%
\pgfpathlineto{\pgfqpoint{1.755682in}{2.267136in}}%
\pgfpathlineto{\pgfqpoint{1.826136in}{2.311704in}}%
\pgfpathlineto{\pgfqpoint{1.896591in}{2.339344in}}%
\pgfpathlineto{\pgfqpoint{1.967045in}{2.369152in}}%
\pgfpathlineto{\pgfqpoint{2.037500in}{2.456305in}}%
\pgfpathlineto{\pgfqpoint{2.107955in}{2.504821in}}%
\pgfpathlineto{\pgfqpoint{2.178409in}{2.553813in}}%
\pgfpathlineto{\pgfqpoint{2.248864in}{2.625794in}}%
\pgfpathlineto{\pgfqpoint{2.319318in}{2.647505in}}%
\pgfpathlineto{\pgfqpoint{2.389773in}{2.672405in}}%
\pgfpathlineto{\pgfqpoint{2.460227in}{2.699162in}}%
\pgfpathlineto{\pgfqpoint{2.530682in}{2.770973in}}%
\pgfpathlineto{\pgfqpoint{2.601136in}{2.765704in}}%
\pgfpathlineto{\pgfqpoint{2.671591in}{2.748295in}}%
\pgfpathlineto{\pgfqpoint{2.742045in}{2.813779in}}%
\pgfpathlineto{\pgfqpoint{2.812500in}{2.821795in}}%
\pgfpathlineto{\pgfqpoint{2.882955in}{2.840666in}}%
\pgfpathlineto{\pgfqpoint{2.953409in}{2.892043in}}%
\pgfpathlineto{\pgfqpoint{3.023864in}{2.877261in}}%
\pgfpathlineto{\pgfqpoint{3.094318in}{2.867940in}}%
\pgfpathlineto{\pgfqpoint{3.164773in}{2.905483in}}%
\pgfpathlineto{\pgfqpoint{3.235227in}{2.949281in}}%
\pgfpathlineto{\pgfqpoint{3.305682in}{2.949569in}}%
\pgfpathlineto{\pgfqpoint{3.376136in}{2.952472in}}%
\pgfpathlineto{\pgfqpoint{3.446591in}{2.948598in}}%
\pgfpathlineto{\pgfqpoint{3.517045in}{2.937731in}}%
\pgfpathlineto{\pgfqpoint{3.587500in}{2.951846in}}%
\pgfpathlineto{\pgfqpoint{3.657955in}{2.970588in}}%
\pgfpathlineto{\pgfqpoint{3.728409in}{3.004887in}}%
\pgfpathlineto{\pgfqpoint{3.798864in}{2.992219in}}%
\pgfpathlineto{\pgfqpoint{3.869318in}{2.987075in}}%
\pgfpathlineto{\pgfqpoint{3.939773in}{2.976487in}}%
\pgfpathlineto{\pgfqpoint{4.010227in}{2.975084in}}%
\pgfpathlineto{\pgfqpoint{4.080682in}{2.985967in}}%
\pgfpathlineto{\pgfqpoint{4.151136in}{3.015867in}}%
\pgfpathlineto{\pgfqpoint{4.221591in}{3.028718in}}%
\pgfpathlineto{\pgfqpoint{4.292045in}{3.007378in}}%
\pgfpathlineto{\pgfqpoint{4.362500in}{3.027002in}}%
\pgfpathlineto{\pgfqpoint{4.432955in}{2.983632in}}%
\pgfpathlineto{\pgfqpoint{4.503409in}{2.983966in}}%
\pgfpathlineto{\pgfqpoint{4.573864in}{3.015288in}}%
\pgfpathlineto{\pgfqpoint{4.644318in}{2.995018in}}%
\pgfpathlineto{\pgfqpoint{4.714773in}{3.009251in}}%
\pgfpathlineto{\pgfqpoint{4.785227in}{3.028653in}}%
\pgfpathlineto{\pgfqpoint{4.855682in}{3.060773in}}%
\pgfpathlineto{\pgfqpoint{4.926136in}{3.043957in}}%
\pgfpathlineto{\pgfqpoint{4.996591in}{3.049558in}}%
\pgfpathlineto{\pgfqpoint{5.067045in}{3.012188in}}%
\pgfpathlineto{\pgfqpoint{5.137500in}{3.001677in}}%
\pgfpathlineto{\pgfqpoint{5.207955in}{3.032481in}}%
\pgfpathlineto{\pgfqpoint{5.278409in}{3.047940in}}%
\pgfpathlineto{\pgfqpoint{5.348864in}{3.064504in}}%
\pgfpathlineto{\pgfqpoint{5.419318in}{3.051880in}}%
\pgfpathlineto{\pgfqpoint{5.489773in}{3.073677in}}%
\pgfpathlineto{\pgfqpoint{5.560227in}{3.030523in}}%
\pgfpathlineto{\pgfqpoint{5.630682in}{3.023026in}}%
\pgfpathlineto{\pgfqpoint{5.701136in}{3.056659in}}%
\pgfpathlineto{\pgfqpoint{5.771591in}{3.070016in}}%
\pgfpathlineto{\pgfqpoint{5.842045in}{3.089052in}}%
\pgfpathlineto{\pgfqpoint{5.912500in}{3.066164in}}%
\pgfpathlineto{\pgfqpoint{5.982955in}{3.077560in}}%
\pgfpathlineto{\pgfqpoint{6.053409in}{3.053684in}}%
\pgfusepath{stroke}%
\end{pgfscope}%
\begin{pgfscope}%
\pgfpathrectangle{\pgfqpoint{0.875000in}{0.440000in}}{\pgfqpoint{5.425000in}{3.080000in}}%
\pgfusepath{clip}%
\pgfsetrectcap%
\pgfsetroundjoin%
\pgfsetlinewidth{1.505625pt}%
\definecolor{currentstroke}{rgb}{0.172549,0.627451,0.172549}%
\pgfsetstrokecolor{currentstroke}%
\pgfsetdash{}{0pt}%
\pgfpathmoveto{\pgfqpoint{1.121591in}{0.580000in}}%
\pgfpathlineto{\pgfqpoint{1.192045in}{0.827240in}}%
\pgfpathlineto{\pgfqpoint{1.262500in}{1.310366in}}%
\pgfpathlineto{\pgfqpoint{1.332955in}{1.610166in}}%
\pgfpathlineto{\pgfqpoint{1.403409in}{1.981464in}}%
\pgfpathlineto{\pgfqpoint{1.473864in}{2.256489in}}%
\pgfpathlineto{\pgfqpoint{1.544318in}{2.372303in}}%
\pgfpathlineto{\pgfqpoint{1.614773in}{2.513904in}}%
\pgfpathlineto{\pgfqpoint{1.685227in}{2.675848in}}%
\pgfpathlineto{\pgfqpoint{1.755682in}{2.776131in}}%
\pgfpathlineto{\pgfqpoint{1.826136in}{2.861417in}}%
\pgfpathlineto{\pgfqpoint{1.896591in}{3.061006in}}%
\pgfpathlineto{\pgfqpoint{1.967045in}{3.145616in}}%
\pgfpathlineto{\pgfqpoint{2.037500in}{3.086026in}}%
\pgfpathlineto{\pgfqpoint{2.107955in}{3.153403in}}%
\pgfpathlineto{\pgfqpoint{2.178409in}{3.246509in}}%
\pgfpathlineto{\pgfqpoint{2.248864in}{3.205408in}}%
\pgfpathlineto{\pgfqpoint{2.319318in}{3.299086in}}%
\pgfpathlineto{\pgfqpoint{2.389773in}{3.248424in}}%
\pgfpathlineto{\pgfqpoint{2.460227in}{3.257331in}}%
\pgfpathlineto{\pgfqpoint{2.530682in}{3.254493in}}%
\pgfpathlineto{\pgfqpoint{2.601136in}{3.314392in}}%
\pgfpathlineto{\pgfqpoint{2.671591in}{3.286881in}}%
\pgfpathlineto{\pgfqpoint{2.742045in}{3.378047in}}%
\pgfpathlineto{\pgfqpoint{2.812500in}{3.335268in}}%
\pgfpathlineto{\pgfqpoint{2.882955in}{3.264539in}}%
\pgfpathlineto{\pgfqpoint{2.953409in}{3.255259in}}%
\pgfpathlineto{\pgfqpoint{3.023864in}{3.293421in}}%
\pgfpathlineto{\pgfqpoint{3.094318in}{3.358147in}}%
\pgfpathlineto{\pgfqpoint{3.164773in}{3.380000in}}%
\pgfpathlineto{\pgfqpoint{3.235227in}{3.339986in}}%
\pgfpathlineto{\pgfqpoint{3.305682in}{3.358612in}}%
\pgfpathlineto{\pgfqpoint{3.376136in}{3.313445in}}%
\pgfpathlineto{\pgfqpoint{3.446591in}{3.308308in}}%
\pgfpathlineto{\pgfqpoint{3.517045in}{3.301298in}}%
\pgfpathlineto{\pgfqpoint{3.587500in}{3.230373in}}%
\pgfpathlineto{\pgfqpoint{3.657955in}{3.311603in}}%
\pgfpathlineto{\pgfqpoint{3.728409in}{3.229024in}}%
\pgfpathlineto{\pgfqpoint{3.798864in}{3.373491in}}%
\pgfpathlineto{\pgfqpoint{3.869318in}{3.253586in}}%
\pgfpathlineto{\pgfqpoint{3.939773in}{3.245615in}}%
\pgfpathlineto{\pgfqpoint{4.010227in}{3.286781in}}%
\pgfpathlineto{\pgfqpoint{4.080682in}{3.205179in}}%
\pgfpathlineto{\pgfqpoint{4.151136in}{3.160198in}}%
\pgfpathlineto{\pgfqpoint{4.221591in}{3.310265in}}%
\pgfpathlineto{\pgfqpoint{4.292045in}{3.349194in}}%
\pgfpathlineto{\pgfqpoint{4.362500in}{3.271256in}}%
\pgfpathlineto{\pgfqpoint{4.432955in}{3.250255in}}%
\pgfpathlineto{\pgfqpoint{4.503409in}{3.295676in}}%
\pgfpathlineto{\pgfqpoint{4.573864in}{3.349530in}}%
\pgfpathlineto{\pgfqpoint{4.644318in}{3.280938in}}%
\pgfpathlineto{\pgfqpoint{4.714773in}{3.250541in}}%
\pgfpathlineto{\pgfqpoint{4.785227in}{3.274227in}}%
\pgfpathlineto{\pgfqpoint{4.855682in}{3.353734in}}%
\pgfpathlineto{\pgfqpoint{4.926136in}{3.305000in}}%
\pgfpathlineto{\pgfqpoint{4.996591in}{3.322218in}}%
\pgfpathlineto{\pgfqpoint{5.067045in}{3.254018in}}%
\pgfpathlineto{\pgfqpoint{5.137500in}{3.282586in}}%
\pgfpathlineto{\pgfqpoint{5.207955in}{3.281588in}}%
\pgfpathlineto{\pgfqpoint{5.278409in}{3.300450in}}%
\pgfpathlineto{\pgfqpoint{5.348864in}{3.348857in}}%
\pgfpathlineto{\pgfqpoint{5.419318in}{3.366208in}}%
\pgfpathlineto{\pgfqpoint{5.489773in}{3.356894in}}%
\pgfpathlineto{\pgfqpoint{5.560227in}{3.274887in}}%
\pgfpathlineto{\pgfqpoint{5.630682in}{3.295904in}}%
\pgfpathlineto{\pgfqpoint{5.701136in}{3.275712in}}%
\pgfpathlineto{\pgfqpoint{5.771591in}{3.338287in}}%
\pgfpathlineto{\pgfqpoint{5.842045in}{3.280660in}}%
\pgfpathlineto{\pgfqpoint{5.912500in}{3.257660in}}%
\pgfpathlineto{\pgfqpoint{5.982955in}{3.319937in}}%
\pgfpathlineto{\pgfqpoint{6.053409in}{3.304038in}}%
\pgfusepath{stroke}%
\end{pgfscope}%
\begin{pgfscope}%
\pgfsetrectcap%
\pgfsetmiterjoin%
\pgfsetlinewidth{0.803000pt}%
\definecolor{currentstroke}{rgb}{0.000000,0.000000,0.000000}%
\pgfsetstrokecolor{currentstroke}%
\pgfsetdash{}{0pt}%
\pgfpathmoveto{\pgfqpoint{0.875000in}{0.440000in}}%
\pgfpathlineto{\pgfqpoint{0.875000in}{3.520000in}}%
\pgfusepath{stroke}%
\end{pgfscope}%
\begin{pgfscope}%
\pgfsetrectcap%
\pgfsetmiterjoin%
\pgfsetlinewidth{0.803000pt}%
\definecolor{currentstroke}{rgb}{0.000000,0.000000,0.000000}%
\pgfsetstrokecolor{currentstroke}%
\pgfsetdash{}{0pt}%
\pgfpathmoveto{\pgfqpoint{6.300000in}{0.440000in}}%
\pgfpathlineto{\pgfqpoint{6.300000in}{3.520000in}}%
\pgfusepath{stroke}%
\end{pgfscope}%
\begin{pgfscope}%
\pgfsetrectcap%
\pgfsetmiterjoin%
\pgfsetlinewidth{0.803000pt}%
\definecolor{currentstroke}{rgb}{0.000000,0.000000,0.000000}%
\pgfsetstrokecolor{currentstroke}%
\pgfsetdash{}{0pt}%
\pgfpathmoveto{\pgfqpoint{0.875000in}{0.440000in}}%
\pgfpathlineto{\pgfqpoint{6.300000in}{0.440000in}}%
\pgfusepath{stroke}%
\end{pgfscope}%
\begin{pgfscope}%
\pgfsetrectcap%
\pgfsetmiterjoin%
\pgfsetlinewidth{0.803000pt}%
\definecolor{currentstroke}{rgb}{0.000000,0.000000,0.000000}%
\pgfsetstrokecolor{currentstroke}%
\pgfsetdash{}{0pt}%
\pgfpathmoveto{\pgfqpoint{0.875000in}{3.520000in}}%
\pgfpathlineto{\pgfqpoint{6.300000in}{3.520000in}}%
\pgfusepath{stroke}%
\end{pgfscope}%
\begin{pgfscope}%
\pgfsetbuttcap%
\pgfsetmiterjoin%
\definecolor{currentfill}{rgb}{1.000000,1.000000,1.000000}%
\pgfsetfillcolor{currentfill}%
\pgfsetfillopacity{0.800000}%
\pgfsetlinewidth{1.003750pt}%
\definecolor{currentstroke}{rgb}{0.800000,0.800000,0.800000}%
\pgfsetstrokecolor{currentstroke}%
\pgfsetstrokeopacity{0.800000}%
\pgfsetdash{}{0pt}%
\pgfpathmoveto{\pgfqpoint{5.124999in}{0.509444in}}%
\pgfpathlineto{\pgfqpoint{6.202778in}{0.509444in}}%
\pgfpathquadraticcurveto{\pgfqpoint{6.230556in}{0.509444in}}{\pgfqpoint{6.230556in}{0.537222in}}%
\pgfpathlineto{\pgfqpoint{6.230556in}{1.104352in}}%
\pgfpathquadraticcurveto{\pgfqpoint{6.230556in}{1.132129in}}{\pgfqpoint{6.202778in}{1.132129in}}%
\pgfpathlineto{\pgfqpoint{5.124999in}{1.132129in}}%
\pgfpathquadraticcurveto{\pgfqpoint{5.097221in}{1.132129in}}{\pgfqpoint{5.097221in}{1.104352in}}%
\pgfpathlineto{\pgfqpoint{5.097221in}{0.537222in}}%
\pgfpathquadraticcurveto{\pgfqpoint{5.097221in}{0.509444in}}{\pgfqpoint{5.124999in}{0.509444in}}%
\pgfpathlineto{\pgfqpoint{5.124999in}{0.509444in}}%
\pgfpathclose%
\pgfusepath{stroke,fill}%
\end{pgfscope}%
\begin{pgfscope}%
\pgfsetrectcap%
\pgfsetroundjoin%
\pgfsetlinewidth{1.505625pt}%
\definecolor{currentstroke}{rgb}{0.121569,0.466667,0.705882}%
\pgfsetstrokecolor{currentstroke}%
\pgfsetdash{}{0pt}%
\pgfpathmoveto{\pgfqpoint{5.152777in}{1.027963in}}%
\pgfpathlineto{\pgfqpoint{5.291666in}{1.027963in}}%
\pgfpathlineto{\pgfqpoint{5.430554in}{1.027963in}}%
\pgfusepath{stroke}%
\end{pgfscope}%
\begin{pgfscope}%
\definecolor{textcolor}{rgb}{0.000000,0.000000,0.000000}%
\pgfsetstrokecolor{textcolor}%
\pgfsetfillcolor{textcolor}%
\pgftext[x=5.541666in,y=0.979352in,left,base]{\color{textcolor}\rmfamily\fontsize{10.000000}{12.000000}\selectfont VC\_Vision}%
\end{pgfscope}%
\begin{pgfscope}%
\pgfsetrectcap%
\pgfsetroundjoin%
\pgfsetlinewidth{1.505625pt}%
\definecolor{currentstroke}{rgb}{1.000000,0.498039,0.054902}%
\pgfsetstrokecolor{currentstroke}%
\pgfsetdash{}{0pt}%
\pgfpathmoveto{\pgfqpoint{5.152777in}{0.834290in}}%
\pgfpathlineto{\pgfqpoint{5.291666in}{0.834290in}}%
\pgfpathlineto{\pgfqpoint{5.430554in}{0.834290in}}%
\pgfusepath{stroke}%
\end{pgfscope}%
\begin{pgfscope}%
\definecolor{textcolor}{rgb}{0.000000,0.000000,0.000000}%
\pgfsetstrokecolor{textcolor}%
\pgfsetfillcolor{textcolor}%
\pgftext[x=5.541666in,y=0.785679in,left,base]{\color{textcolor}\rmfamily\fontsize{10.000000}{12.000000}\selectfont VC\_Proj}%
\end{pgfscope}%
\begin{pgfscope}%
\pgfsetrectcap%
\pgfsetroundjoin%
\pgfsetlinewidth{1.505625pt}%
\definecolor{currentstroke}{rgb}{0.172549,0.627451,0.172549}%
\pgfsetstrokecolor{currentstroke}%
\pgfsetdash{}{0pt}%
\pgfpathmoveto{\pgfqpoint{5.152777in}{0.640617in}}%
\pgfpathlineto{\pgfqpoint{5.291666in}{0.640617in}}%
\pgfpathlineto{\pgfqpoint{5.430554in}{0.640617in}}%
\pgfusepath{stroke}%
\end{pgfscope}%
\begin{pgfscope}%
\definecolor{textcolor}{rgb}{0.000000,0.000000,0.000000}%
\pgfsetstrokecolor{textcolor}%
\pgfsetfillcolor{textcolor}%
\pgftext[x=5.541666in,y=0.592006in,left,base]{\color{textcolor}\rmfamily\fontsize{10.000000}{12.000000}\selectfont IC}%
\end{pgfscope}%
\end{pgfpicture}%
\makeatother%
\endgroup%
}
    \caption[mAP performance on each epoch for VC\_Vision, VC\_Proj, and IC]{This chart illustrates the mAP performance of the models on each epoch.}
    \label{fig:tp_backbone}
\end{figure}

\subsection{Does Text Embedding Help with the Long-Tail Issue?}
In this experiment, I want to investigate whether text embedding assists in addressing the long-tail issue. This can be done by comparing the performance of each segment of the IC and CARe models. The IC model aims to output a class embedding as its learning target, while the CARe model outputs one-hot encoding results. From Figure \ref{fig:tp_longtailcomp}, it is obvious that the majority of the performance improvement comes from the middle and tail classes. The performance improvement in the head classes is only 6.2\%, while the improvements in the middle and tail classes are 21.9\% and 20.7\%, respectively. This result indicates that text embedding does indeed help with the long-tail issue. 

\begin{figure}[ht]
    \centering
    \resizebox{1.0\textwidth}{!}{%% Creator: Matplotlib, PGF backend
%%
%% To include the figure in your LaTeX document, write
%%   \input{<filename>.pgf}
%%
%% Make sure the required packages are loaded in your preamble
%%   \usepackage{pgf}
%%
%% and, on pdftex
%%   \usepackage[utf8]{inputenc}\DeclareUnicodeCharacter{2212}{-}
%%
%% or, on luatex and xetex
%%   \usepackage{unicode-math}
%%
%% Figures using additional raster images can only be included by \input if
%% they are in the same directory as the main LaTeX file. For loading figures
%% from other directories you can use the `import` package
%%   \usepackage{import}
%%
%% and then include the figures with
%%   \import{<path to file>}{<filename>.pgf}
%%
%% Matplotlib used the following preamble
%%
\begingroup%
\makeatletter%
\begin{pgfpicture}%
\pgfpathrectangle{\pgfpointorigin}{\pgfqpoint{7.000000in}{5.000000in}}%
\pgfusepath{use as bounding box, clip}%
\begin{pgfscope}%
\pgfsetbuttcap%
\pgfsetmiterjoin%
\definecolor{currentfill}{rgb}{1.000000,1.000000,1.000000}%
\pgfsetfillcolor{currentfill}%
\pgfsetlinewidth{0.000000pt}%
\definecolor{currentstroke}{rgb}{1.000000,1.000000,1.000000}%
\pgfsetstrokecolor{currentstroke}%
\pgfsetdash{}{0pt}%
\pgfpathmoveto{\pgfqpoint{0.000000in}{0.000000in}}%
\pgfpathlineto{\pgfqpoint{7.000000in}{0.000000in}}%
\pgfpathlineto{\pgfqpoint{7.000000in}{5.000000in}}%
\pgfpathlineto{\pgfqpoint{0.000000in}{5.000000in}}%
\pgfpathclose%
\pgfusepath{fill}%
\end{pgfscope}%
\begin{pgfscope}%
\pgfsetbuttcap%
\pgfsetmiterjoin%
\definecolor{currentfill}{rgb}{1.000000,1.000000,1.000000}%
\pgfsetfillcolor{currentfill}%
\pgfsetlinewidth{0.000000pt}%
\definecolor{currentstroke}{rgb}{0.000000,0.000000,0.000000}%
\pgfsetstrokecolor{currentstroke}%
\pgfsetstrokeopacity{0.000000}%
\pgfsetdash{}{0pt}%
\pgfpathmoveto{\pgfqpoint{0.565124in}{0.549691in}}%
\pgfpathlineto{\pgfqpoint{6.850000in}{0.549691in}}%
\pgfpathlineto{\pgfqpoint{6.850000in}{4.650926in}}%
\pgfpathlineto{\pgfqpoint{0.565124in}{4.650926in}}%
\pgfpathclose%
\pgfusepath{fill}%
\end{pgfscope}%
\begin{pgfscope}%
\pgfpathrectangle{\pgfqpoint{0.565124in}{0.549691in}}{\pgfqpoint{6.284876in}{4.101235in}}%
\pgfusepath{clip}%
\pgfsetbuttcap%
\pgfsetmiterjoin%
\definecolor{currentfill}{rgb}{0.121569,0.466667,0.705882}%
\pgfsetfillcolor{currentfill}%
\pgfsetlinewidth{0.000000pt}%
\definecolor{currentstroke}{rgb}{0.000000,0.000000,0.000000}%
\pgfsetstrokecolor{currentstroke}%
\pgfsetstrokeopacity{0.000000}%
\pgfsetdash{}{0pt}%
\pgfpathmoveto{\pgfqpoint{0.850800in}{0.549691in}}%
\pgfpathlineto{\pgfqpoint{1.422152in}{0.549691in}}%
\pgfpathlineto{\pgfqpoint{1.422152in}{4.107336in}}%
\pgfpathlineto{\pgfqpoint{0.850800in}{4.107336in}}%
\pgfpathclose%
\pgfusepath{fill}%
\end{pgfscope}%
\begin{pgfscope}%
\pgfpathrectangle{\pgfqpoint{0.565124in}{0.549691in}}{\pgfqpoint{6.284876in}{4.101235in}}%
\pgfusepath{clip}%
\pgfsetbuttcap%
\pgfsetmiterjoin%
\definecolor{currentfill}{rgb}{0.121569,0.466667,0.705882}%
\pgfsetfillcolor{currentfill}%
\pgfsetlinewidth{0.000000pt}%
\definecolor{currentstroke}{rgb}{0.000000,0.000000,0.000000}%
\pgfsetstrokecolor{currentstroke}%
\pgfsetstrokeopacity{0.000000}%
\pgfsetdash{}{0pt}%
\pgfpathmoveto{\pgfqpoint{3.136209in}{0.549691in}}%
\pgfpathlineto{\pgfqpoint{3.707562in}{0.549691in}}%
\pgfpathlineto{\pgfqpoint{3.707562in}{2.719220in}}%
\pgfpathlineto{\pgfqpoint{3.136209in}{2.719220in}}%
\pgfpathclose%
\pgfusepath{fill}%
\end{pgfscope}%
\begin{pgfscope}%
\pgfpathrectangle{\pgfqpoint{0.565124in}{0.549691in}}{\pgfqpoint{6.284876in}{4.101235in}}%
\pgfusepath{clip}%
\pgfsetbuttcap%
\pgfsetmiterjoin%
\definecolor{currentfill}{rgb}{0.121569,0.466667,0.705882}%
\pgfsetfillcolor{currentfill}%
\pgfsetlinewidth{0.000000pt}%
\definecolor{currentstroke}{rgb}{0.000000,0.000000,0.000000}%
\pgfsetstrokecolor{currentstroke}%
\pgfsetstrokeopacity{0.000000}%
\pgfsetdash{}{0pt}%
\pgfpathmoveto{\pgfqpoint{5.421619in}{0.549691in}}%
\pgfpathlineto{\pgfqpoint{5.992971in}{0.549691in}}%
\pgfpathlineto{\pgfqpoint{5.992971in}{1.959155in}}%
\pgfpathlineto{\pgfqpoint{5.421619in}{1.959155in}}%
\pgfpathclose%
\pgfusepath{fill}%
\end{pgfscope}%
\begin{pgfscope}%
\pgfpathrectangle{\pgfqpoint{0.565124in}{0.549691in}}{\pgfqpoint{6.284876in}{4.101235in}}%
\pgfusepath{clip}%
\pgfsetbuttcap%
\pgfsetmiterjoin%
\definecolor{currentfill}{rgb}{1.000000,0.498039,0.054902}%
\pgfsetfillcolor{currentfill}%
\pgfsetlinewidth{0.000000pt}%
\definecolor{currentstroke}{rgb}{0.000000,0.000000,0.000000}%
\pgfsetstrokecolor{currentstroke}%
\pgfsetstrokeopacity{0.000000}%
\pgfsetdash{}{0pt}%
\pgfpathmoveto{\pgfqpoint{1.422152in}{0.549691in}}%
\pgfpathlineto{\pgfqpoint{1.993505in}{0.549691in}}%
\pgfpathlineto{\pgfqpoint{1.993505in}{4.455629in}}%
\pgfpathlineto{\pgfqpoint{1.422152in}{4.455629in}}%
\pgfpathclose%
\pgfusepath{fill}%
\end{pgfscope}%
\begin{pgfscope}%
\pgfpathrectangle{\pgfqpoint{0.565124in}{0.549691in}}{\pgfqpoint{6.284876in}{4.101235in}}%
\pgfusepath{clip}%
\pgfsetbuttcap%
\pgfsetmiterjoin%
\definecolor{currentfill}{rgb}{1.000000,0.498039,0.054902}%
\pgfsetfillcolor{currentfill}%
\pgfsetlinewidth{0.000000pt}%
\definecolor{currentstroke}{rgb}{0.000000,0.000000,0.000000}%
\pgfsetstrokecolor{currentstroke}%
\pgfsetstrokeopacity{0.000000}%
\pgfsetdash{}{0pt}%
\pgfpathmoveto{\pgfqpoint{3.707562in}{0.549691in}}%
\pgfpathlineto{\pgfqpoint{4.278914in}{0.549691in}}%
\pgfpathlineto{\pgfqpoint{4.278914in}{3.949481in}}%
\pgfpathlineto{\pgfqpoint{3.707562in}{3.949481in}}%
\pgfpathclose%
\pgfusepath{fill}%
\end{pgfscope}%
\begin{pgfscope}%
\pgfpathrectangle{\pgfqpoint{0.565124in}{0.549691in}}{\pgfqpoint{6.284876in}{4.101235in}}%
\pgfusepath{clip}%
\pgfsetbuttcap%
\pgfsetmiterjoin%
\definecolor{currentfill}{rgb}{1.000000,0.498039,0.054902}%
\pgfsetfillcolor{currentfill}%
\pgfsetlinewidth{0.000000pt}%
\definecolor{currentstroke}{rgb}{0.000000,0.000000,0.000000}%
\pgfsetstrokecolor{currentstroke}%
\pgfsetstrokeopacity{0.000000}%
\pgfsetdash{}{0pt}%
\pgfpathmoveto{\pgfqpoint{5.992971in}{0.549691in}}%
\pgfpathlineto{\pgfqpoint{6.564324in}{0.549691in}}%
\pgfpathlineto{\pgfqpoint{6.564324in}{3.122004in}}%
\pgfpathlineto{\pgfqpoint{5.992971in}{3.122004in}}%
\pgfpathclose%
\pgfusepath{fill}%
\end{pgfscope}%
\begin{pgfscope}%
\pgfsetbuttcap%
\pgfsetroundjoin%
\definecolor{currentfill}{rgb}{0.000000,0.000000,0.000000}%
\pgfsetfillcolor{currentfill}%
\pgfsetlinewidth{0.803000pt}%
\definecolor{currentstroke}{rgb}{0.000000,0.000000,0.000000}%
\pgfsetstrokecolor{currentstroke}%
\pgfsetdash{}{0pt}%
\pgfsys@defobject{currentmarker}{\pgfqpoint{0.000000in}{-0.048611in}}{\pgfqpoint{0.000000in}{0.000000in}}{%
\pgfpathmoveto{\pgfqpoint{0.000000in}{0.000000in}}%
\pgfpathlineto{\pgfqpoint{0.000000in}{-0.048611in}}%
\pgfusepath{stroke,fill}%
}%
\begin{pgfscope}%
\pgfsys@transformshift{1.422152in}{0.549691in}%
\pgfsys@useobject{currentmarker}{}%
\end{pgfscope}%
\end{pgfscope}%
\begin{pgfscope}%
\definecolor{textcolor}{rgb}{0.000000,0.000000,0.000000}%
\pgfsetstrokecolor{textcolor}%
\pgfsetfillcolor{textcolor}%
\pgftext[x=1.422152in,y=0.452469in,,top]{\color{textcolor}\rmfamily\fontsize{10.000000}{12.000000}\selectfont Head}%
\end{pgfscope}%
\begin{pgfscope}%
\pgfsetbuttcap%
\pgfsetroundjoin%
\definecolor{currentfill}{rgb}{0.000000,0.000000,0.000000}%
\pgfsetfillcolor{currentfill}%
\pgfsetlinewidth{0.803000pt}%
\definecolor{currentstroke}{rgb}{0.000000,0.000000,0.000000}%
\pgfsetstrokecolor{currentstroke}%
\pgfsetdash{}{0pt}%
\pgfsys@defobject{currentmarker}{\pgfqpoint{0.000000in}{-0.048611in}}{\pgfqpoint{0.000000in}{0.000000in}}{%
\pgfpathmoveto{\pgfqpoint{0.000000in}{0.000000in}}%
\pgfpathlineto{\pgfqpoint{0.000000in}{-0.048611in}}%
\pgfusepath{stroke,fill}%
}%
\begin{pgfscope}%
\pgfsys@transformshift{3.707562in}{0.549691in}%
\pgfsys@useobject{currentmarker}{}%
\end{pgfscope}%
\end{pgfscope}%
\begin{pgfscope}%
\definecolor{textcolor}{rgb}{0.000000,0.000000,0.000000}%
\pgfsetstrokecolor{textcolor}%
\pgfsetfillcolor{textcolor}%
\pgftext[x=3.707562in,y=0.452469in,,top]{\color{textcolor}\rmfamily\fontsize{10.000000}{12.000000}\selectfont Middle}%
\end{pgfscope}%
\begin{pgfscope}%
\pgfsetbuttcap%
\pgfsetroundjoin%
\definecolor{currentfill}{rgb}{0.000000,0.000000,0.000000}%
\pgfsetfillcolor{currentfill}%
\pgfsetlinewidth{0.803000pt}%
\definecolor{currentstroke}{rgb}{0.000000,0.000000,0.000000}%
\pgfsetstrokecolor{currentstroke}%
\pgfsetdash{}{0pt}%
\pgfsys@defobject{currentmarker}{\pgfqpoint{0.000000in}{-0.048611in}}{\pgfqpoint{0.000000in}{0.000000in}}{%
\pgfpathmoveto{\pgfqpoint{0.000000in}{0.000000in}}%
\pgfpathlineto{\pgfqpoint{0.000000in}{-0.048611in}}%
\pgfusepath{stroke,fill}%
}%
\begin{pgfscope}%
\pgfsys@transformshift{5.992971in}{0.549691in}%
\pgfsys@useobject{currentmarker}{}%
\end{pgfscope}%
\end{pgfscope}%
\begin{pgfscope}%
\definecolor{textcolor}{rgb}{0.000000,0.000000,0.000000}%
\pgfsetstrokecolor{textcolor}%
\pgfsetfillcolor{textcolor}%
\pgftext[x=5.992971in,y=0.452469in,,top]{\color{textcolor}\rmfamily\fontsize{10.000000}{12.000000}\selectfont Tail}%
\end{pgfscope}%
\begin{pgfscope}%
\definecolor{textcolor}{rgb}{0.000000,0.000000,0.000000}%
\pgfsetstrokecolor{textcolor}%
\pgfsetfillcolor{textcolor}%
\pgftext[x=3.707562in,y=0.273457in,,top]{\color{textcolor}\rmfamily\fontsize{10.000000}{12.000000}\selectfont Segments}%
\end{pgfscope}%
\begin{pgfscope}%
\pgfsetbuttcap%
\pgfsetroundjoin%
\definecolor{currentfill}{rgb}{0.000000,0.000000,0.000000}%
\pgfsetfillcolor{currentfill}%
\pgfsetlinewidth{0.803000pt}%
\definecolor{currentstroke}{rgb}{0.000000,0.000000,0.000000}%
\pgfsetstrokecolor{currentstroke}%
\pgfsetdash{}{0pt}%
\pgfsys@defobject{currentmarker}{\pgfqpoint{-0.048611in}{0.000000in}}{\pgfqpoint{0.000000in}{0.000000in}}{%
\pgfpathmoveto{\pgfqpoint{0.000000in}{0.000000in}}%
\pgfpathlineto{\pgfqpoint{-0.048611in}{0.000000in}}%
\pgfusepath{stroke,fill}%
}%
\begin{pgfscope}%
\pgfsys@transformshift{0.565124in}{0.549691in}%
\pgfsys@useobject{currentmarker}{}%
\end{pgfscope}%
\end{pgfscope}%
\begin{pgfscope}%
\definecolor{textcolor}{rgb}{0.000000,0.000000,0.000000}%
\pgfsetstrokecolor{textcolor}%
\pgfsetfillcolor{textcolor}%
\pgftext[x=0.398457in, y=0.501466in, left, base]{\color{textcolor}\rmfamily\fontsize{10.000000}{12.000000}\selectfont \(\displaystyle {0}\)}%
\end{pgfscope}%
\begin{pgfscope}%
\pgfsetbuttcap%
\pgfsetroundjoin%
\definecolor{currentfill}{rgb}{0.000000,0.000000,0.000000}%
\pgfsetfillcolor{currentfill}%
\pgfsetlinewidth{0.803000pt}%
\definecolor{currentstroke}{rgb}{0.000000,0.000000,0.000000}%
\pgfsetstrokecolor{currentstroke}%
\pgfsetdash{}{0pt}%
\pgfsys@defobject{currentmarker}{\pgfqpoint{-0.048611in}{0.000000in}}{\pgfqpoint{0.000000in}{0.000000in}}{%
\pgfpathmoveto{\pgfqpoint{0.000000in}{0.000000in}}%
\pgfpathlineto{\pgfqpoint{-0.048611in}{0.000000in}}%
\pgfusepath{stroke,fill}%
}%
\begin{pgfscope}%
\pgfsys@transformshift{0.565124in}{1.111454in}%
\pgfsys@useobject{currentmarker}{}%
\end{pgfscope}%
\end{pgfscope}%
\begin{pgfscope}%
\definecolor{textcolor}{rgb}{0.000000,0.000000,0.000000}%
\pgfsetstrokecolor{textcolor}%
\pgfsetfillcolor{textcolor}%
\pgftext[x=0.329012in, y=1.063229in, left, base]{\color{textcolor}\rmfamily\fontsize{10.000000}{12.000000}\selectfont \(\displaystyle {10}\)}%
\end{pgfscope}%
\begin{pgfscope}%
\pgfsetbuttcap%
\pgfsetroundjoin%
\definecolor{currentfill}{rgb}{0.000000,0.000000,0.000000}%
\pgfsetfillcolor{currentfill}%
\pgfsetlinewidth{0.803000pt}%
\definecolor{currentstroke}{rgb}{0.000000,0.000000,0.000000}%
\pgfsetstrokecolor{currentstroke}%
\pgfsetdash{}{0pt}%
\pgfsys@defobject{currentmarker}{\pgfqpoint{-0.048611in}{0.000000in}}{\pgfqpoint{0.000000in}{0.000000in}}{%
\pgfpathmoveto{\pgfqpoint{0.000000in}{0.000000in}}%
\pgfpathlineto{\pgfqpoint{-0.048611in}{0.000000in}}%
\pgfusepath{stroke,fill}%
}%
\begin{pgfscope}%
\pgfsys@transformshift{0.565124in}{1.673217in}%
\pgfsys@useobject{currentmarker}{}%
\end{pgfscope}%
\end{pgfscope}%
\begin{pgfscope}%
\definecolor{textcolor}{rgb}{0.000000,0.000000,0.000000}%
\pgfsetstrokecolor{textcolor}%
\pgfsetfillcolor{textcolor}%
\pgftext[x=0.329012in, y=1.624992in, left, base]{\color{textcolor}\rmfamily\fontsize{10.000000}{12.000000}\selectfont \(\displaystyle {20}\)}%
\end{pgfscope}%
\begin{pgfscope}%
\pgfsetbuttcap%
\pgfsetroundjoin%
\definecolor{currentfill}{rgb}{0.000000,0.000000,0.000000}%
\pgfsetfillcolor{currentfill}%
\pgfsetlinewidth{0.803000pt}%
\definecolor{currentstroke}{rgb}{0.000000,0.000000,0.000000}%
\pgfsetstrokecolor{currentstroke}%
\pgfsetdash{}{0pt}%
\pgfsys@defobject{currentmarker}{\pgfqpoint{-0.048611in}{0.000000in}}{\pgfqpoint{0.000000in}{0.000000in}}{%
\pgfpathmoveto{\pgfqpoint{0.000000in}{0.000000in}}%
\pgfpathlineto{\pgfqpoint{-0.048611in}{0.000000in}}%
\pgfusepath{stroke,fill}%
}%
\begin{pgfscope}%
\pgfsys@transformshift{0.565124in}{2.234980in}%
\pgfsys@useobject{currentmarker}{}%
\end{pgfscope}%
\end{pgfscope}%
\begin{pgfscope}%
\definecolor{textcolor}{rgb}{0.000000,0.000000,0.000000}%
\pgfsetstrokecolor{textcolor}%
\pgfsetfillcolor{textcolor}%
\pgftext[x=0.329012in, y=2.186755in, left, base]{\color{textcolor}\rmfamily\fontsize{10.000000}{12.000000}\selectfont \(\displaystyle {30}\)}%
\end{pgfscope}%
\begin{pgfscope}%
\pgfsetbuttcap%
\pgfsetroundjoin%
\definecolor{currentfill}{rgb}{0.000000,0.000000,0.000000}%
\pgfsetfillcolor{currentfill}%
\pgfsetlinewidth{0.803000pt}%
\definecolor{currentstroke}{rgb}{0.000000,0.000000,0.000000}%
\pgfsetstrokecolor{currentstroke}%
\pgfsetdash{}{0pt}%
\pgfsys@defobject{currentmarker}{\pgfqpoint{-0.048611in}{0.000000in}}{\pgfqpoint{0.000000in}{0.000000in}}{%
\pgfpathmoveto{\pgfqpoint{0.000000in}{0.000000in}}%
\pgfpathlineto{\pgfqpoint{-0.048611in}{0.000000in}}%
\pgfusepath{stroke,fill}%
}%
\begin{pgfscope}%
\pgfsys@transformshift{0.565124in}{2.796743in}%
\pgfsys@useobject{currentmarker}{}%
\end{pgfscope}%
\end{pgfscope}%
\begin{pgfscope}%
\definecolor{textcolor}{rgb}{0.000000,0.000000,0.000000}%
\pgfsetstrokecolor{textcolor}%
\pgfsetfillcolor{textcolor}%
\pgftext[x=0.329012in, y=2.748518in, left, base]{\color{textcolor}\rmfamily\fontsize{10.000000}{12.000000}\selectfont \(\displaystyle {40}\)}%
\end{pgfscope}%
\begin{pgfscope}%
\pgfsetbuttcap%
\pgfsetroundjoin%
\definecolor{currentfill}{rgb}{0.000000,0.000000,0.000000}%
\pgfsetfillcolor{currentfill}%
\pgfsetlinewidth{0.803000pt}%
\definecolor{currentstroke}{rgb}{0.000000,0.000000,0.000000}%
\pgfsetstrokecolor{currentstroke}%
\pgfsetdash{}{0pt}%
\pgfsys@defobject{currentmarker}{\pgfqpoint{-0.048611in}{0.000000in}}{\pgfqpoint{0.000000in}{0.000000in}}{%
\pgfpathmoveto{\pgfqpoint{0.000000in}{0.000000in}}%
\pgfpathlineto{\pgfqpoint{-0.048611in}{0.000000in}}%
\pgfusepath{stroke,fill}%
}%
\begin{pgfscope}%
\pgfsys@transformshift{0.565124in}{3.358506in}%
\pgfsys@useobject{currentmarker}{}%
\end{pgfscope}%
\end{pgfscope}%
\begin{pgfscope}%
\definecolor{textcolor}{rgb}{0.000000,0.000000,0.000000}%
\pgfsetstrokecolor{textcolor}%
\pgfsetfillcolor{textcolor}%
\pgftext[x=0.329012in, y=3.310281in, left, base]{\color{textcolor}\rmfamily\fontsize{10.000000}{12.000000}\selectfont \(\displaystyle {50}\)}%
\end{pgfscope}%
\begin{pgfscope}%
\pgfsetbuttcap%
\pgfsetroundjoin%
\definecolor{currentfill}{rgb}{0.000000,0.000000,0.000000}%
\pgfsetfillcolor{currentfill}%
\pgfsetlinewidth{0.803000pt}%
\definecolor{currentstroke}{rgb}{0.000000,0.000000,0.000000}%
\pgfsetstrokecolor{currentstroke}%
\pgfsetdash{}{0pt}%
\pgfsys@defobject{currentmarker}{\pgfqpoint{-0.048611in}{0.000000in}}{\pgfqpoint{0.000000in}{0.000000in}}{%
\pgfpathmoveto{\pgfqpoint{0.000000in}{0.000000in}}%
\pgfpathlineto{\pgfqpoint{-0.048611in}{0.000000in}}%
\pgfusepath{stroke,fill}%
}%
\begin{pgfscope}%
\pgfsys@transformshift{0.565124in}{3.920269in}%
\pgfsys@useobject{currentmarker}{}%
\end{pgfscope}%
\end{pgfscope}%
\begin{pgfscope}%
\definecolor{textcolor}{rgb}{0.000000,0.000000,0.000000}%
\pgfsetstrokecolor{textcolor}%
\pgfsetfillcolor{textcolor}%
\pgftext[x=0.329012in, y=3.872044in, left, base]{\color{textcolor}\rmfamily\fontsize{10.000000}{12.000000}\selectfont \(\displaystyle {60}\)}%
\end{pgfscope}%
\begin{pgfscope}%
\pgfsetbuttcap%
\pgfsetroundjoin%
\definecolor{currentfill}{rgb}{0.000000,0.000000,0.000000}%
\pgfsetfillcolor{currentfill}%
\pgfsetlinewidth{0.803000pt}%
\definecolor{currentstroke}{rgb}{0.000000,0.000000,0.000000}%
\pgfsetstrokecolor{currentstroke}%
\pgfsetdash{}{0pt}%
\pgfsys@defobject{currentmarker}{\pgfqpoint{-0.048611in}{0.000000in}}{\pgfqpoint{0.000000in}{0.000000in}}{%
\pgfpathmoveto{\pgfqpoint{0.000000in}{0.000000in}}%
\pgfpathlineto{\pgfqpoint{-0.048611in}{0.000000in}}%
\pgfusepath{stroke,fill}%
}%
\begin{pgfscope}%
\pgfsys@transformshift{0.565124in}{4.482032in}%
\pgfsys@useobject{currentmarker}{}%
\end{pgfscope}%
\end{pgfscope}%
\begin{pgfscope}%
\definecolor{textcolor}{rgb}{0.000000,0.000000,0.000000}%
\pgfsetstrokecolor{textcolor}%
\pgfsetfillcolor{textcolor}%
\pgftext[x=0.329012in, y=4.433807in, left, base]{\color{textcolor}\rmfamily\fontsize{10.000000}{12.000000}\selectfont \(\displaystyle {70}\)}%
\end{pgfscope}%
\begin{pgfscope}%
\definecolor{textcolor}{rgb}{0.000000,0.000000,0.000000}%
\pgfsetstrokecolor{textcolor}%
\pgfsetfillcolor{textcolor}%
\pgftext[x=0.273457in,y=2.600309in,,bottom,rotate=90.000000]{\color{textcolor}\rmfamily\fontsize{10.000000}{12.000000}\selectfont mAP Score}%
\end{pgfscope}%
\begin{pgfscope}%
\pgfsetrectcap%
\pgfsetmiterjoin%
\pgfsetlinewidth{0.803000pt}%
\definecolor{currentstroke}{rgb}{0.000000,0.000000,0.000000}%
\pgfsetstrokecolor{currentstroke}%
\pgfsetdash{}{0pt}%
\pgfpathmoveto{\pgfqpoint{0.565124in}{0.549691in}}%
\pgfpathlineto{\pgfqpoint{0.565124in}{4.650926in}}%
\pgfusepath{stroke}%
\end{pgfscope}%
\begin{pgfscope}%
\pgfsetrectcap%
\pgfsetmiterjoin%
\pgfsetlinewidth{0.803000pt}%
\definecolor{currentstroke}{rgb}{0.000000,0.000000,0.000000}%
\pgfsetstrokecolor{currentstroke}%
\pgfsetdash{}{0pt}%
\pgfpathmoveto{\pgfqpoint{6.850000in}{0.549691in}}%
\pgfpathlineto{\pgfqpoint{6.850000in}{4.650926in}}%
\pgfusepath{stroke}%
\end{pgfscope}%
\begin{pgfscope}%
\pgfsetrectcap%
\pgfsetmiterjoin%
\pgfsetlinewidth{0.803000pt}%
\definecolor{currentstroke}{rgb}{0.000000,0.000000,0.000000}%
\pgfsetstrokecolor{currentstroke}%
\pgfsetdash{}{0pt}%
\pgfpathmoveto{\pgfqpoint{0.565124in}{0.549691in}}%
\pgfpathlineto{\pgfqpoint{6.850000in}{0.549691in}}%
\pgfusepath{stroke}%
\end{pgfscope}%
\begin{pgfscope}%
\pgfsetrectcap%
\pgfsetmiterjoin%
\pgfsetlinewidth{0.803000pt}%
\definecolor{currentstroke}{rgb}{0.000000,0.000000,0.000000}%
\pgfsetstrokecolor{currentstroke}%
\pgfsetdash{}{0pt}%
\pgfpathmoveto{\pgfqpoint{0.565124in}{4.650926in}}%
\pgfpathlineto{\pgfqpoint{6.850000in}{4.650926in}}%
\pgfusepath{stroke}%
\end{pgfscope}%
\begin{pgfscope}%
\pgfsetroundcap%
\pgfsetroundjoin%
\definecolor{currentfill}{rgb}{0.000000,0.000000,0.000000}%
\pgfsetfillcolor{currentfill}%
\pgfsetlinewidth{1.003750pt}%
\definecolor{currentstroke}{rgb}{0.000000,0.000000,0.000000}%
\pgfsetstrokecolor{currentstroke}%
\pgfsetdash{}{0pt}%
\pgfpathmoveto{\pgfqpoint{1.177811in}{4.311784in}}%
\pgfpathquadraticcurveto{\pgfqpoint{1.268107in}{4.281838in}}{\pgfqpoint{1.358403in}{4.251892in}}%
\pgfpathlineto{\pgfqpoint{1.367147in}{4.278258in}}%
\pgfpathquadraticcurveto{\pgfqpoint{1.394650in}{4.250846in}}{\pgfqpoint{1.422152in}{4.223434in}}%
\pgfpathquadraticcurveto{\pgfqpoint{1.383720in}{4.217889in}}{\pgfqpoint{1.345287in}{4.212344in}}%
\pgfpathlineto{\pgfqpoint{1.354031in}{4.238709in}}%
\pgfpathquadraticcurveto{\pgfqpoint{1.263735in}{4.268655in}}{\pgfqpoint{1.173439in}{4.298601in}}%
\pgfpathlineto{\pgfqpoint{1.177811in}{4.311784in}}%
\pgfpathclose%
\pgfusepath{stroke,fill}%
\end{pgfscope}%
\begin{pgfscope}%
\definecolor{textcolor}{rgb}{1.000000,0.000000,0.000000}%
\pgfsetstrokecolor{textcolor}%
\pgfsetfillcolor{textcolor}%
\pgftext[x=0.850800in, y=4.484151in, left, base]{\color{textcolor}\rmfamily\fontsize{10.000000}{12.000000}\selectfont diff: }%
\end{pgfscope}%
\begin{pgfscope}%
\definecolor{textcolor}{rgb}{1.000000,0.000000,0.000000}%
\pgfsetstrokecolor{textcolor}%
\pgfsetfillcolor{textcolor}%
\pgftext[x=0.850800in, y=4.341404in, left, base]{\color{textcolor}\rmfamily\fontsize{10.000000}{12.000000}\selectfont 6.20}%
\end{pgfscope}%
\begin{pgfscope}%
\pgfsetroundcap%
\pgfsetroundjoin%
\definecolor{currentfill}{rgb}{0.000000,0.000000,0.000000}%
\pgfsetfillcolor{currentfill}%
\pgfsetlinewidth{1.003750pt}%
\definecolor{currentstroke}{rgb}{0.000000,0.000000,0.000000}%
\pgfsetstrokecolor{currentstroke}%
\pgfsetdash{}{0pt}%
\pgfpathmoveto{\pgfqpoint{3.139704in}{3.324417in}}%
\pgfpathquadraticcurveto{\pgfqpoint{3.391758in}{3.241060in}}{\pgfqpoint{3.643811in}{3.157704in}}%
\pgfpathlineto{\pgfqpoint{3.652533in}{3.184077in}}%
\pgfpathquadraticcurveto{\pgfqpoint{3.680047in}{3.156692in}}{\pgfqpoint{3.707562in}{3.129307in}}%
\pgfpathquadraticcurveto{\pgfqpoint{3.669145in}{3.123726in}}{\pgfqpoint{3.630728in}{3.118145in}}%
\pgfpathlineto{\pgfqpoint{3.639450in}{3.144518in}}%
\pgfpathquadraticcurveto{\pgfqpoint{3.387397in}{3.227874in}}{\pgfqpoint{3.135343in}{3.311230in}}%
\pgfpathlineto{\pgfqpoint{3.139704in}{3.324417in}}%
\pgfpathclose%
\pgfusepath{stroke,fill}%
\end{pgfscope}%
\begin{pgfscope}%
\definecolor{textcolor}{rgb}{1.000000,0.000000,0.000000}%
\pgfsetstrokecolor{textcolor}%
\pgfsetfillcolor{textcolor}%
\pgftext[x=2.793398in, y=3.496759in, left, base]{\color{textcolor}\rmfamily\fontsize{10.000000}{12.000000}\selectfont diff: }%
\end{pgfscope}%
\begin{pgfscope}%
\definecolor{textcolor}{rgb}{1.000000,0.000000,0.000000}%
\pgfsetstrokecolor{textcolor}%
\pgfsetfillcolor{textcolor}%
\pgftext[x=2.793398in, y=3.354012in, left, base]{\color{textcolor}\rmfamily\fontsize{10.000000}{12.000000}\selectfont 21.90}%
\end{pgfscope}%
\begin{pgfscope}%
\pgfsetroundcap%
\pgfsetroundjoin%
\definecolor{currentfill}{rgb}{0.000000,0.000000,0.000000}%
\pgfsetfillcolor{currentfill}%
\pgfsetlinewidth{1.003750pt}%
\definecolor{currentstroke}{rgb}{0.000000,0.000000,0.000000}%
\pgfsetstrokecolor{currentstroke}%
\pgfsetdash{}{0pt}%
\pgfpathmoveto{\pgfqpoint{5.425114in}{2.541881in}}%
\pgfpathquadraticcurveto{\pgfqpoint{5.677167in}{2.458525in}}{\pgfqpoint{5.929220in}{2.375168in}}%
\pgfpathlineto{\pgfqpoint{5.937942in}{2.401541in}}%
\pgfpathquadraticcurveto{\pgfqpoint{5.965457in}{2.374156in}}{\pgfqpoint{5.992971in}{2.346771in}}%
\pgfpathquadraticcurveto{\pgfqpoint{5.954555in}{2.341190in}}{\pgfqpoint{5.916138in}{2.335609in}}%
\pgfpathlineto{\pgfqpoint{5.924859in}{2.361982in}}%
\pgfpathquadraticcurveto{\pgfqpoint{5.672806in}{2.445338in}}{\pgfqpoint{5.420753in}{2.528694in}}%
\pgfpathlineto{\pgfqpoint{5.425114in}{2.541881in}}%
\pgfpathclose%
\pgfusepath{stroke,fill}%
\end{pgfscope}%
\begin{pgfscope}%
\definecolor{textcolor}{rgb}{1.000000,0.000000,0.000000}%
\pgfsetstrokecolor{textcolor}%
\pgfsetfillcolor{textcolor}%
\pgftext[x=5.078808in, y=2.714223in, left, base]{\color{textcolor}\rmfamily\fontsize{10.000000}{12.000000}\selectfont diff: }%
\end{pgfscope}%
\begin{pgfscope}%
\definecolor{textcolor}{rgb}{1.000000,0.000000,0.000000}%
\pgfsetstrokecolor{textcolor}%
\pgfsetfillcolor{textcolor}%
\pgftext[x=5.078808in, y=2.571476in, left, base]{\color{textcolor}\rmfamily\fontsize{10.000000}{12.000000}\selectfont 20.70}%
\end{pgfscope}%
\begin{pgfscope}%
\definecolor{textcolor}{rgb}{0.000000,0.000000,0.000000}%
\pgfsetstrokecolor{textcolor}%
\pgfsetfillcolor{textcolor}%
\pgftext[x=3.707562in,y=4.734260in,,base]{\color{textcolor}\rmfamily\fontsize{12.000000}{14.400000}\selectfont Comparison of mAP scores for CARe and IC models}%
\end{pgfscope}%
\begin{pgfscope}%
\pgfsetbuttcap%
\pgfsetmiterjoin%
\definecolor{currentfill}{rgb}{1.000000,1.000000,1.000000}%
\pgfsetfillcolor{currentfill}%
\pgfsetfillopacity{0.800000}%
\pgfsetlinewidth{1.003750pt}%
\definecolor{currentstroke}{rgb}{0.800000,0.800000,0.800000}%
\pgfsetstrokecolor{currentstroke}%
\pgfsetstrokeopacity{0.800000}%
\pgfsetdash{}{0pt}%
\pgfpathmoveto{\pgfqpoint{5.939892in}{4.152470in}}%
\pgfpathlineto{\pgfqpoint{6.752778in}{4.152470in}}%
\pgfpathquadraticcurveto{\pgfqpoint{6.780556in}{4.152470in}}{\pgfqpoint{6.780556in}{4.180247in}}%
\pgfpathlineto{\pgfqpoint{6.780556in}{4.553704in}}%
\pgfpathquadraticcurveto{\pgfqpoint{6.780556in}{4.581482in}}{\pgfqpoint{6.752778in}{4.581482in}}%
\pgfpathlineto{\pgfqpoint{5.939892in}{4.581482in}}%
\pgfpathquadraticcurveto{\pgfqpoint{5.912114in}{4.581482in}}{\pgfqpoint{5.912114in}{4.553704in}}%
\pgfpathlineto{\pgfqpoint{5.912114in}{4.180247in}}%
\pgfpathquadraticcurveto{\pgfqpoint{5.912114in}{4.152470in}}{\pgfqpoint{5.939892in}{4.152470in}}%
\pgfpathclose%
\pgfusepath{stroke,fill}%
\end{pgfscope}%
\begin{pgfscope}%
\pgfsetbuttcap%
\pgfsetmiterjoin%
\definecolor{currentfill}{rgb}{0.121569,0.466667,0.705882}%
\pgfsetfillcolor{currentfill}%
\pgfsetlinewidth{0.000000pt}%
\definecolor{currentstroke}{rgb}{0.000000,0.000000,0.000000}%
\pgfsetstrokecolor{currentstroke}%
\pgfsetstrokeopacity{0.000000}%
\pgfsetdash{}{0pt}%
\pgfpathmoveto{\pgfqpoint{5.967669in}{4.428704in}}%
\pgfpathlineto{\pgfqpoint{6.245447in}{4.428704in}}%
\pgfpathlineto{\pgfqpoint{6.245447in}{4.525926in}}%
\pgfpathlineto{\pgfqpoint{5.967669in}{4.525926in}}%
\pgfpathclose%
\pgfusepath{fill}%
\end{pgfscope}%
\begin{pgfscope}%
\definecolor{textcolor}{rgb}{0.000000,0.000000,0.000000}%
\pgfsetstrokecolor{textcolor}%
\pgfsetfillcolor{textcolor}%
\pgftext[x=6.356558in,y=4.428704in,left,base]{\color{textcolor}\rmfamily\fontsize{10.000000}{12.000000}\selectfont CARe}%
\end{pgfscope}%
\begin{pgfscope}%
\pgfsetbuttcap%
\pgfsetmiterjoin%
\definecolor{currentfill}{rgb}{1.000000,0.498039,0.054902}%
\pgfsetfillcolor{currentfill}%
\pgfsetlinewidth{0.000000pt}%
\definecolor{currentstroke}{rgb}{0.000000,0.000000,0.000000}%
\pgfsetstrokecolor{currentstroke}%
\pgfsetstrokeopacity{0.000000}%
\pgfsetdash{}{0pt}%
\pgfpathmoveto{\pgfqpoint{5.967669in}{4.235031in}}%
\pgfpathlineto{\pgfqpoint{6.245447in}{4.235031in}}%
\pgfpathlineto{\pgfqpoint{6.245447in}{4.332253in}}%
\pgfpathlineto{\pgfqpoint{5.967669in}{4.332253in}}%
\pgfpathclose%
\pgfusepath{fill}%
\end{pgfscope}%
\begin{pgfscope}%
\definecolor{textcolor}{rgb}{0.000000,0.000000,0.000000}%
\pgfsetstrokecolor{textcolor}%
\pgfsetfillcolor{textcolor}%
\pgftext[x=6.356558in,y=4.235031in,left,base]{\color{textcolor}\rmfamily\fontsize{10.000000}{12.000000}\selectfont IC}%
\end{pgfscope}%
\end{pgfpicture}%
\makeatother%
\endgroup%
}
    \caption[mAP performance on the best epoch for CARe and IC]{This chart illustrates the performance differences for each segment of actions for the CARe and IC models.}
    \label{fig:tp_longtailcomp}
\end{figure}

% In adddition, to investigate further. I experiment the following two model settings: 

% \begin{enumerate}
%     \item IC: Image Clip with learning target of class embedding
%     \item IC\_onehot: Image Clip with learning target of onehot encoding
% \end{enumerate}






\section{AFRICAN}
\subsection{AFRICAN Pretraining}
% \subsubsection{Training Result}
The model is trained with infoNCE \parencite{oord2019representation} loss, using the adamw optimizer with a learning rate of 0.00003. In each batch, the model processes one video as input, which is then sampled into 8 frames for embedding and similarity matrix computation. The model is trained for 50 epochs on a single NVIDIA A100 GPU. As contrastive learning can be defined as a binary classification task, accuracy serves as the evaluation metric.

To assess the impact of AFRICAN pretraining, I compare the performance of the CLIP model with and without AFRICAN pretraining, denoted as "with-AFRICAN" and "w/o-AFRICAN", respectively. The results are presented in Table \ref{tab:africanpretrainingresults}, and the epoch-wise performance is visualized in Figure \ref{fig:tp_africanpretraining}. The findings suggest that the model is able to distinguish the identity source of augmented images with almost 80\% accuracy. 

\begin{table}[ht]
    \centering
    \caption{Training Results for AFRICAN Pretraining}
    \label{tab:africanpretrainingresults}
    \begin{tabular}{lllll}
        \toprule
        \multirow{2}{*}{Models} & \multicolumn{2}{c}{Accuracy} \\
        \cmidrule{2-3} 
        {} &  Best & Epoch 50\\
        \midrule
        w/o-AFRICAN   & 30.36 & 30.36 \\
        with-AFRICAN  & 79.29 & 78.19 \\
        \bottomrule
    \end{tabular}
\end{table}

\begin{figure}[ht]
    \centering
    \resizebox{1.0\textwidth}{!}{%% Creator: Matplotlib, PGF backend
%%
%% To include the figure in your LaTeX document, write
%%   \input{<filename>.pgf}
%%
%% Make sure the required packages are loaded in your preamble
%%   \usepackage{pgf}
%%
%% and, on pdftex
%%   \usepackage[utf8]{inputenc}\DeclareUnicodeCharacter{2212}{-}
%%
%% or, on luatex and xetex
%%   \usepackage{unicode-math}
%%
%% Figures using additional raster images can only be included by \input if
%% they are in the same directory as the main LaTeX file. For loading figures
%% from other directories you can use the `import` package
%%   \usepackage{import}
%%
%% and then include the figures with
%%   \import{<path to file>}{<filename>.pgf}
%%
%% Matplotlib used the following preamble
%%
\begingroup%
\makeatletter%
\begin{pgfpicture}%
\pgfpathrectangle{\pgfpointorigin}{\pgfqpoint{7.000000in}{4.000000in}}%
\pgfusepath{use as bounding box, clip}%
\begin{pgfscope}%
\pgfsetbuttcap%
\pgfsetmiterjoin%
\definecolor{currentfill}{rgb}{1.000000,1.000000,1.000000}%
\pgfsetfillcolor{currentfill}%
\pgfsetlinewidth{0.000000pt}%
\definecolor{currentstroke}{rgb}{1.000000,1.000000,1.000000}%
\pgfsetstrokecolor{currentstroke}%
\pgfsetdash{}{0pt}%
\pgfpathmoveto{\pgfqpoint{0.000000in}{0.000000in}}%
\pgfpathlineto{\pgfqpoint{7.000000in}{0.000000in}}%
\pgfpathlineto{\pgfqpoint{7.000000in}{4.000000in}}%
\pgfpathlineto{\pgfqpoint{0.000000in}{4.000000in}}%
\pgfpathclose%
\pgfusepath{fill}%
\end{pgfscope}%
\begin{pgfscope}%
\pgfsetbuttcap%
\pgfsetmiterjoin%
\definecolor{currentfill}{rgb}{1.000000,1.000000,1.000000}%
\pgfsetfillcolor{currentfill}%
\pgfsetlinewidth{0.000000pt}%
\definecolor{currentstroke}{rgb}{0.000000,0.000000,0.000000}%
\pgfsetstrokecolor{currentstroke}%
\pgfsetstrokeopacity{0.000000}%
\pgfsetdash{}{0pt}%
\pgfpathmoveto{\pgfqpoint{0.875000in}{0.440000in}}%
\pgfpathlineto{\pgfqpoint{6.300000in}{0.440000in}}%
\pgfpathlineto{\pgfqpoint{6.300000in}{3.520000in}}%
\pgfpathlineto{\pgfqpoint{0.875000in}{3.520000in}}%
\pgfpathclose%
\pgfusepath{fill}%
\end{pgfscope}%
\begin{pgfscope}%
\pgfpathrectangle{\pgfqpoint{0.875000in}{0.440000in}}{\pgfqpoint{5.425000in}{3.080000in}}%
\pgfusepath{clip}%
\pgfsetrectcap%
\pgfsetroundjoin%
\pgfsetlinewidth{0.803000pt}%
\definecolor{currentstroke}{rgb}{0.690196,0.690196,0.690196}%
\pgfsetstrokecolor{currentstroke}%
\pgfsetdash{}{0pt}%
\pgfpathmoveto{\pgfqpoint{1.121591in}{0.440000in}}%
\pgfpathlineto{\pgfqpoint{1.121591in}{3.520000in}}%
\pgfusepath{stroke}%
\end{pgfscope}%
\begin{pgfscope}%
\pgfsetbuttcap%
\pgfsetroundjoin%
\definecolor{currentfill}{rgb}{0.000000,0.000000,0.000000}%
\pgfsetfillcolor{currentfill}%
\pgfsetlinewidth{0.803000pt}%
\definecolor{currentstroke}{rgb}{0.000000,0.000000,0.000000}%
\pgfsetstrokecolor{currentstroke}%
\pgfsetdash{}{0pt}%
\pgfsys@defobject{currentmarker}{\pgfqpoint{0.000000in}{-0.048611in}}{\pgfqpoint{0.000000in}{0.000000in}}{%
\pgfpathmoveto{\pgfqpoint{0.000000in}{0.000000in}}%
\pgfpathlineto{\pgfqpoint{0.000000in}{-0.048611in}}%
\pgfusepath{stroke,fill}%
}%
\begin{pgfscope}%
\pgfsys@transformshift{1.121591in}{0.440000in}%
\pgfsys@useobject{currentmarker}{}%
\end{pgfscope}%
\end{pgfscope}%
\begin{pgfscope}%
\definecolor{textcolor}{rgb}{0.000000,0.000000,0.000000}%
\pgfsetstrokecolor{textcolor}%
\pgfsetfillcolor{textcolor}%
\pgftext[x=1.121591in,y=0.342778in,,top]{\color{textcolor}\rmfamily\fontsize{10.000000}{12.000000}\selectfont \(\displaystyle {0}\)}%
\end{pgfscope}%
\begin{pgfscope}%
\pgfpathrectangle{\pgfqpoint{0.875000in}{0.440000in}}{\pgfqpoint{5.425000in}{3.080000in}}%
\pgfusepath{clip}%
\pgfsetrectcap%
\pgfsetroundjoin%
\pgfsetlinewidth{0.803000pt}%
\definecolor{currentstroke}{rgb}{0.690196,0.690196,0.690196}%
\pgfsetstrokecolor{currentstroke}%
\pgfsetdash{}{0pt}%
\pgfpathmoveto{\pgfqpoint{2.107955in}{0.440000in}}%
\pgfpathlineto{\pgfqpoint{2.107955in}{3.520000in}}%
\pgfusepath{stroke}%
\end{pgfscope}%
\begin{pgfscope}%
\pgfsetbuttcap%
\pgfsetroundjoin%
\definecolor{currentfill}{rgb}{0.000000,0.000000,0.000000}%
\pgfsetfillcolor{currentfill}%
\pgfsetlinewidth{0.803000pt}%
\definecolor{currentstroke}{rgb}{0.000000,0.000000,0.000000}%
\pgfsetstrokecolor{currentstroke}%
\pgfsetdash{}{0pt}%
\pgfsys@defobject{currentmarker}{\pgfqpoint{0.000000in}{-0.048611in}}{\pgfqpoint{0.000000in}{0.000000in}}{%
\pgfpathmoveto{\pgfqpoint{0.000000in}{0.000000in}}%
\pgfpathlineto{\pgfqpoint{0.000000in}{-0.048611in}}%
\pgfusepath{stroke,fill}%
}%
\begin{pgfscope}%
\pgfsys@transformshift{2.107955in}{0.440000in}%
\pgfsys@useobject{currentmarker}{}%
\end{pgfscope}%
\end{pgfscope}%
\begin{pgfscope}%
\definecolor{textcolor}{rgb}{0.000000,0.000000,0.000000}%
\pgfsetstrokecolor{textcolor}%
\pgfsetfillcolor{textcolor}%
\pgftext[x=2.107955in,y=0.342778in,,top]{\color{textcolor}\rmfamily\fontsize{10.000000}{12.000000}\selectfont \(\displaystyle {10}\)}%
\end{pgfscope}%
\begin{pgfscope}%
\pgfpathrectangle{\pgfqpoint{0.875000in}{0.440000in}}{\pgfqpoint{5.425000in}{3.080000in}}%
\pgfusepath{clip}%
\pgfsetrectcap%
\pgfsetroundjoin%
\pgfsetlinewidth{0.803000pt}%
\definecolor{currentstroke}{rgb}{0.690196,0.690196,0.690196}%
\pgfsetstrokecolor{currentstroke}%
\pgfsetdash{}{0pt}%
\pgfpathmoveto{\pgfqpoint{3.094318in}{0.440000in}}%
\pgfpathlineto{\pgfqpoint{3.094318in}{3.520000in}}%
\pgfusepath{stroke}%
\end{pgfscope}%
\begin{pgfscope}%
\pgfsetbuttcap%
\pgfsetroundjoin%
\definecolor{currentfill}{rgb}{0.000000,0.000000,0.000000}%
\pgfsetfillcolor{currentfill}%
\pgfsetlinewidth{0.803000pt}%
\definecolor{currentstroke}{rgb}{0.000000,0.000000,0.000000}%
\pgfsetstrokecolor{currentstroke}%
\pgfsetdash{}{0pt}%
\pgfsys@defobject{currentmarker}{\pgfqpoint{0.000000in}{-0.048611in}}{\pgfqpoint{0.000000in}{0.000000in}}{%
\pgfpathmoveto{\pgfqpoint{0.000000in}{0.000000in}}%
\pgfpathlineto{\pgfqpoint{0.000000in}{-0.048611in}}%
\pgfusepath{stroke,fill}%
}%
\begin{pgfscope}%
\pgfsys@transformshift{3.094318in}{0.440000in}%
\pgfsys@useobject{currentmarker}{}%
\end{pgfscope}%
\end{pgfscope}%
\begin{pgfscope}%
\definecolor{textcolor}{rgb}{0.000000,0.000000,0.000000}%
\pgfsetstrokecolor{textcolor}%
\pgfsetfillcolor{textcolor}%
\pgftext[x=3.094318in,y=0.342778in,,top]{\color{textcolor}\rmfamily\fontsize{10.000000}{12.000000}\selectfont \(\displaystyle {20}\)}%
\end{pgfscope}%
\begin{pgfscope}%
\pgfpathrectangle{\pgfqpoint{0.875000in}{0.440000in}}{\pgfqpoint{5.425000in}{3.080000in}}%
\pgfusepath{clip}%
\pgfsetrectcap%
\pgfsetroundjoin%
\pgfsetlinewidth{0.803000pt}%
\definecolor{currentstroke}{rgb}{0.690196,0.690196,0.690196}%
\pgfsetstrokecolor{currentstroke}%
\pgfsetdash{}{0pt}%
\pgfpathmoveto{\pgfqpoint{4.080682in}{0.440000in}}%
\pgfpathlineto{\pgfqpoint{4.080682in}{3.520000in}}%
\pgfusepath{stroke}%
\end{pgfscope}%
\begin{pgfscope}%
\pgfsetbuttcap%
\pgfsetroundjoin%
\definecolor{currentfill}{rgb}{0.000000,0.000000,0.000000}%
\pgfsetfillcolor{currentfill}%
\pgfsetlinewidth{0.803000pt}%
\definecolor{currentstroke}{rgb}{0.000000,0.000000,0.000000}%
\pgfsetstrokecolor{currentstroke}%
\pgfsetdash{}{0pt}%
\pgfsys@defobject{currentmarker}{\pgfqpoint{0.000000in}{-0.048611in}}{\pgfqpoint{0.000000in}{0.000000in}}{%
\pgfpathmoveto{\pgfqpoint{0.000000in}{0.000000in}}%
\pgfpathlineto{\pgfqpoint{0.000000in}{-0.048611in}}%
\pgfusepath{stroke,fill}%
}%
\begin{pgfscope}%
\pgfsys@transformshift{4.080682in}{0.440000in}%
\pgfsys@useobject{currentmarker}{}%
\end{pgfscope}%
\end{pgfscope}%
\begin{pgfscope}%
\definecolor{textcolor}{rgb}{0.000000,0.000000,0.000000}%
\pgfsetstrokecolor{textcolor}%
\pgfsetfillcolor{textcolor}%
\pgftext[x=4.080682in,y=0.342778in,,top]{\color{textcolor}\rmfamily\fontsize{10.000000}{12.000000}\selectfont \(\displaystyle {30}\)}%
\end{pgfscope}%
\begin{pgfscope}%
\pgfpathrectangle{\pgfqpoint{0.875000in}{0.440000in}}{\pgfqpoint{5.425000in}{3.080000in}}%
\pgfusepath{clip}%
\pgfsetrectcap%
\pgfsetroundjoin%
\pgfsetlinewidth{0.803000pt}%
\definecolor{currentstroke}{rgb}{0.690196,0.690196,0.690196}%
\pgfsetstrokecolor{currentstroke}%
\pgfsetdash{}{0pt}%
\pgfpathmoveto{\pgfqpoint{5.067045in}{0.440000in}}%
\pgfpathlineto{\pgfqpoint{5.067045in}{3.520000in}}%
\pgfusepath{stroke}%
\end{pgfscope}%
\begin{pgfscope}%
\pgfsetbuttcap%
\pgfsetroundjoin%
\definecolor{currentfill}{rgb}{0.000000,0.000000,0.000000}%
\pgfsetfillcolor{currentfill}%
\pgfsetlinewidth{0.803000pt}%
\definecolor{currentstroke}{rgb}{0.000000,0.000000,0.000000}%
\pgfsetstrokecolor{currentstroke}%
\pgfsetdash{}{0pt}%
\pgfsys@defobject{currentmarker}{\pgfqpoint{0.000000in}{-0.048611in}}{\pgfqpoint{0.000000in}{0.000000in}}{%
\pgfpathmoveto{\pgfqpoint{0.000000in}{0.000000in}}%
\pgfpathlineto{\pgfqpoint{0.000000in}{-0.048611in}}%
\pgfusepath{stroke,fill}%
}%
\begin{pgfscope}%
\pgfsys@transformshift{5.067045in}{0.440000in}%
\pgfsys@useobject{currentmarker}{}%
\end{pgfscope}%
\end{pgfscope}%
\begin{pgfscope}%
\definecolor{textcolor}{rgb}{0.000000,0.000000,0.000000}%
\pgfsetstrokecolor{textcolor}%
\pgfsetfillcolor{textcolor}%
\pgftext[x=5.067045in,y=0.342778in,,top]{\color{textcolor}\rmfamily\fontsize{10.000000}{12.000000}\selectfont \(\displaystyle {40}\)}%
\end{pgfscope}%
\begin{pgfscope}%
\pgfpathrectangle{\pgfqpoint{0.875000in}{0.440000in}}{\pgfqpoint{5.425000in}{3.080000in}}%
\pgfusepath{clip}%
\pgfsetrectcap%
\pgfsetroundjoin%
\pgfsetlinewidth{0.803000pt}%
\definecolor{currentstroke}{rgb}{0.690196,0.690196,0.690196}%
\pgfsetstrokecolor{currentstroke}%
\pgfsetdash{}{0pt}%
\pgfpathmoveto{\pgfqpoint{6.053409in}{0.440000in}}%
\pgfpathlineto{\pgfqpoint{6.053409in}{3.520000in}}%
\pgfusepath{stroke}%
\end{pgfscope}%
\begin{pgfscope}%
\pgfsetbuttcap%
\pgfsetroundjoin%
\definecolor{currentfill}{rgb}{0.000000,0.000000,0.000000}%
\pgfsetfillcolor{currentfill}%
\pgfsetlinewidth{0.803000pt}%
\definecolor{currentstroke}{rgb}{0.000000,0.000000,0.000000}%
\pgfsetstrokecolor{currentstroke}%
\pgfsetdash{}{0pt}%
\pgfsys@defobject{currentmarker}{\pgfqpoint{0.000000in}{-0.048611in}}{\pgfqpoint{0.000000in}{0.000000in}}{%
\pgfpathmoveto{\pgfqpoint{0.000000in}{0.000000in}}%
\pgfpathlineto{\pgfqpoint{0.000000in}{-0.048611in}}%
\pgfusepath{stroke,fill}%
}%
\begin{pgfscope}%
\pgfsys@transformshift{6.053409in}{0.440000in}%
\pgfsys@useobject{currentmarker}{}%
\end{pgfscope}%
\end{pgfscope}%
\begin{pgfscope}%
\definecolor{textcolor}{rgb}{0.000000,0.000000,0.000000}%
\pgfsetstrokecolor{textcolor}%
\pgfsetfillcolor{textcolor}%
\pgftext[x=6.053409in,y=0.342778in,,top]{\color{textcolor}\rmfamily\fontsize{10.000000}{12.000000}\selectfont \(\displaystyle {50}\)}%
\end{pgfscope}%
\begin{pgfscope}%
\definecolor{textcolor}{rgb}{0.000000,0.000000,0.000000}%
\pgfsetstrokecolor{textcolor}%
\pgfsetfillcolor{textcolor}%
\pgftext[x=3.587500in,y=0.163766in,,top]{\color{textcolor}\rmfamily\fontsize{10.000000}{12.000000}\selectfont Epoch}%
\end{pgfscope}%
\begin{pgfscope}%
\pgfpathrectangle{\pgfqpoint{0.875000in}{0.440000in}}{\pgfqpoint{5.425000in}{3.080000in}}%
\pgfusepath{clip}%
\pgfsetrectcap%
\pgfsetroundjoin%
\pgfsetlinewidth{0.803000pt}%
\definecolor{currentstroke}{rgb}{0.690196,0.690196,0.690196}%
\pgfsetstrokecolor{currentstroke}%
\pgfsetdash{}{0pt}%
\pgfpathmoveto{\pgfqpoint{0.875000in}{1.075006in}}%
\pgfpathlineto{\pgfqpoint{6.300000in}{1.075006in}}%
\pgfusepath{stroke}%
\end{pgfscope}%
\begin{pgfscope}%
\pgfsetbuttcap%
\pgfsetroundjoin%
\definecolor{currentfill}{rgb}{0.000000,0.000000,0.000000}%
\pgfsetfillcolor{currentfill}%
\pgfsetlinewidth{0.803000pt}%
\definecolor{currentstroke}{rgb}{0.000000,0.000000,0.000000}%
\pgfsetstrokecolor{currentstroke}%
\pgfsetdash{}{0pt}%
\pgfsys@defobject{currentmarker}{\pgfqpoint{-0.048611in}{0.000000in}}{\pgfqpoint{0.000000in}{0.000000in}}{%
\pgfpathmoveto{\pgfqpoint{0.000000in}{0.000000in}}%
\pgfpathlineto{\pgfqpoint{-0.048611in}{0.000000in}}%
\pgfusepath{stroke,fill}%
}%
\begin{pgfscope}%
\pgfsys@transformshift{0.875000in}{1.075006in}%
\pgfsys@useobject{currentmarker}{}%
\end{pgfscope}%
\end{pgfscope}%
\begin{pgfscope}%
\definecolor{textcolor}{rgb}{0.000000,0.000000,0.000000}%
\pgfsetstrokecolor{textcolor}%
\pgfsetfillcolor{textcolor}%
\pgftext[x=0.530863in, y=1.026781in, left, base]{\color{textcolor}\rmfamily\fontsize{10.000000}{12.000000}\selectfont \(\displaystyle {0.72}\)}%
\end{pgfscope}%
\begin{pgfscope}%
\pgfpathrectangle{\pgfqpoint{0.875000in}{0.440000in}}{\pgfqpoint{5.425000in}{3.080000in}}%
\pgfusepath{clip}%
\pgfsetrectcap%
\pgfsetroundjoin%
\pgfsetlinewidth{0.803000pt}%
\definecolor{currentstroke}{rgb}{0.690196,0.690196,0.690196}%
\pgfsetstrokecolor{currentstroke}%
\pgfsetdash{}{0pt}%
\pgfpathmoveto{\pgfqpoint{0.875000in}{1.759412in}}%
\pgfpathlineto{\pgfqpoint{6.300000in}{1.759412in}}%
\pgfusepath{stroke}%
\end{pgfscope}%
\begin{pgfscope}%
\pgfsetbuttcap%
\pgfsetroundjoin%
\definecolor{currentfill}{rgb}{0.000000,0.000000,0.000000}%
\pgfsetfillcolor{currentfill}%
\pgfsetlinewidth{0.803000pt}%
\definecolor{currentstroke}{rgb}{0.000000,0.000000,0.000000}%
\pgfsetstrokecolor{currentstroke}%
\pgfsetdash{}{0pt}%
\pgfsys@defobject{currentmarker}{\pgfqpoint{-0.048611in}{0.000000in}}{\pgfqpoint{0.000000in}{0.000000in}}{%
\pgfpathmoveto{\pgfqpoint{0.000000in}{0.000000in}}%
\pgfpathlineto{\pgfqpoint{-0.048611in}{0.000000in}}%
\pgfusepath{stroke,fill}%
}%
\begin{pgfscope}%
\pgfsys@transformshift{0.875000in}{1.759412in}%
\pgfsys@useobject{currentmarker}{}%
\end{pgfscope}%
\end{pgfscope}%
\begin{pgfscope}%
\definecolor{textcolor}{rgb}{0.000000,0.000000,0.000000}%
\pgfsetstrokecolor{textcolor}%
\pgfsetfillcolor{textcolor}%
\pgftext[x=0.530863in, y=1.711186in, left, base]{\color{textcolor}\rmfamily\fontsize{10.000000}{12.000000}\selectfont \(\displaystyle {0.74}\)}%
\end{pgfscope}%
\begin{pgfscope}%
\pgfpathrectangle{\pgfqpoint{0.875000in}{0.440000in}}{\pgfqpoint{5.425000in}{3.080000in}}%
\pgfusepath{clip}%
\pgfsetrectcap%
\pgfsetroundjoin%
\pgfsetlinewidth{0.803000pt}%
\definecolor{currentstroke}{rgb}{0.690196,0.690196,0.690196}%
\pgfsetstrokecolor{currentstroke}%
\pgfsetdash{}{0pt}%
\pgfpathmoveto{\pgfqpoint{0.875000in}{2.443817in}}%
\pgfpathlineto{\pgfqpoint{6.300000in}{2.443817in}}%
\pgfusepath{stroke}%
\end{pgfscope}%
\begin{pgfscope}%
\pgfsetbuttcap%
\pgfsetroundjoin%
\definecolor{currentfill}{rgb}{0.000000,0.000000,0.000000}%
\pgfsetfillcolor{currentfill}%
\pgfsetlinewidth{0.803000pt}%
\definecolor{currentstroke}{rgb}{0.000000,0.000000,0.000000}%
\pgfsetstrokecolor{currentstroke}%
\pgfsetdash{}{0pt}%
\pgfsys@defobject{currentmarker}{\pgfqpoint{-0.048611in}{0.000000in}}{\pgfqpoint{0.000000in}{0.000000in}}{%
\pgfpathmoveto{\pgfqpoint{0.000000in}{0.000000in}}%
\pgfpathlineto{\pgfqpoint{-0.048611in}{0.000000in}}%
\pgfusepath{stroke,fill}%
}%
\begin{pgfscope}%
\pgfsys@transformshift{0.875000in}{2.443817in}%
\pgfsys@useobject{currentmarker}{}%
\end{pgfscope}%
\end{pgfscope}%
\begin{pgfscope}%
\definecolor{textcolor}{rgb}{0.000000,0.000000,0.000000}%
\pgfsetstrokecolor{textcolor}%
\pgfsetfillcolor{textcolor}%
\pgftext[x=0.530863in, y=2.395592in, left, base]{\color{textcolor}\rmfamily\fontsize{10.000000}{12.000000}\selectfont \(\displaystyle {0.76}\)}%
\end{pgfscope}%
\begin{pgfscope}%
\pgfpathrectangle{\pgfqpoint{0.875000in}{0.440000in}}{\pgfqpoint{5.425000in}{3.080000in}}%
\pgfusepath{clip}%
\pgfsetrectcap%
\pgfsetroundjoin%
\pgfsetlinewidth{0.803000pt}%
\definecolor{currentstroke}{rgb}{0.690196,0.690196,0.690196}%
\pgfsetstrokecolor{currentstroke}%
\pgfsetdash{}{0pt}%
\pgfpathmoveto{\pgfqpoint{0.875000in}{3.128222in}}%
\pgfpathlineto{\pgfqpoint{6.300000in}{3.128222in}}%
\pgfusepath{stroke}%
\end{pgfscope}%
\begin{pgfscope}%
\pgfsetbuttcap%
\pgfsetroundjoin%
\definecolor{currentfill}{rgb}{0.000000,0.000000,0.000000}%
\pgfsetfillcolor{currentfill}%
\pgfsetlinewidth{0.803000pt}%
\definecolor{currentstroke}{rgb}{0.000000,0.000000,0.000000}%
\pgfsetstrokecolor{currentstroke}%
\pgfsetdash{}{0pt}%
\pgfsys@defobject{currentmarker}{\pgfqpoint{-0.048611in}{0.000000in}}{\pgfqpoint{0.000000in}{0.000000in}}{%
\pgfpathmoveto{\pgfqpoint{0.000000in}{0.000000in}}%
\pgfpathlineto{\pgfqpoint{-0.048611in}{0.000000in}}%
\pgfusepath{stroke,fill}%
}%
\begin{pgfscope}%
\pgfsys@transformshift{0.875000in}{3.128222in}%
\pgfsys@useobject{currentmarker}{}%
\end{pgfscope}%
\end{pgfscope}%
\begin{pgfscope}%
\definecolor{textcolor}{rgb}{0.000000,0.000000,0.000000}%
\pgfsetstrokecolor{textcolor}%
\pgfsetfillcolor{textcolor}%
\pgftext[x=0.530863in, y=3.079997in, left, base]{\color{textcolor}\rmfamily\fontsize{10.000000}{12.000000}\selectfont \(\displaystyle {0.78}\)}%
\end{pgfscope}%
\begin{pgfscope}%
\definecolor{textcolor}{rgb}{0.000000,0.000000,0.000000}%
\pgfsetstrokecolor{textcolor}%
\pgfsetfillcolor{textcolor}%
\pgftext[x=0.475308in,y=1.980000in,,bottom,rotate=90.000000]{\color{textcolor}\rmfamily\fontsize{10.000000}{12.000000}\selectfont Accuracy}%
\end{pgfscope}%
\begin{pgfscope}%
\pgfpathrectangle{\pgfqpoint{0.875000in}{0.440000in}}{\pgfqpoint{5.425000in}{3.080000in}}%
\pgfusepath{clip}%
\pgfsetrectcap%
\pgfsetroundjoin%
\pgfsetlinewidth{1.505625pt}%
\definecolor{currentstroke}{rgb}{0.121569,0.466667,0.705882}%
\pgfsetstrokecolor{currentstroke}%
\pgfsetdash{}{0pt}%
\pgfpathmoveto{\pgfqpoint{1.121591in}{0.580000in}}%
\pgfpathlineto{\pgfqpoint{1.220227in}{1.590032in}}%
\pgfpathlineto{\pgfqpoint{1.318864in}{2.180297in}}%
\pgfpathlineto{\pgfqpoint{1.417500in}{2.451159in}}%
\pgfpathlineto{\pgfqpoint{1.516136in}{2.032812in}}%
\pgfpathlineto{\pgfqpoint{1.614773in}{2.489224in}}%
\pgfpathlineto{\pgfqpoint{1.713409in}{2.560409in}}%
\pgfpathlineto{\pgfqpoint{1.812045in}{2.704323in}}%
\pgfpathlineto{\pgfqpoint{1.910682in}{2.833690in}}%
\pgfpathlineto{\pgfqpoint{2.009318in}{2.777717in}}%
\pgfpathlineto{\pgfqpoint{2.107955in}{3.172072in}}%
\pgfpathlineto{\pgfqpoint{2.206591in}{3.034956in}}%
\pgfpathlineto{\pgfqpoint{2.305227in}{2.788142in}}%
\pgfpathlineto{\pgfqpoint{2.403864in}{3.108644in}}%
\pgfpathlineto{\pgfqpoint{2.502500in}{3.066618in}}%
\pgfpathlineto{\pgfqpoint{2.601136in}{2.757198in}}%
\pgfpathlineto{\pgfqpoint{2.699773in}{2.703999in}}%
\pgfpathlineto{\pgfqpoint{2.798409in}{2.851966in}}%
\pgfpathlineto{\pgfqpoint{2.897045in}{2.619266in}}%
\pgfpathlineto{\pgfqpoint{2.995682in}{2.704256in}}%
\pgfpathlineto{\pgfqpoint{3.094318in}{2.894873in}}%
\pgfpathlineto{\pgfqpoint{3.192955in}{3.108954in}}%
\pgfpathlineto{\pgfqpoint{3.291591in}{3.182989in}}%
\pgfpathlineto{\pgfqpoint{3.390227in}{3.042256in}}%
\pgfpathlineto{\pgfqpoint{3.488864in}{3.013338in}}%
\pgfpathlineto{\pgfqpoint{3.587500in}{2.918460in}}%
\pgfpathlineto{\pgfqpoint{3.686136in}{2.904826in}}%
\pgfpathlineto{\pgfqpoint{3.784773in}{2.964985in}}%
\pgfpathlineto{\pgfqpoint{3.883409in}{3.249052in}}%
\pgfpathlineto{\pgfqpoint{3.982045in}{3.380000in}}%
\pgfpathlineto{\pgfqpoint{4.080682in}{3.258025in}}%
\pgfpathlineto{\pgfqpoint{4.179318in}{3.077274in}}%
\pgfpathlineto{\pgfqpoint{4.277955in}{3.105727in}}%
\pgfpathlineto{\pgfqpoint{4.376591in}{2.981643in}}%
\pgfpathlineto{\pgfqpoint{4.475227in}{3.031573in}}%
\pgfpathlineto{\pgfqpoint{4.573864in}{3.122261in}}%
\pgfpathlineto{\pgfqpoint{4.672500in}{3.211520in}}%
\pgfpathlineto{\pgfqpoint{4.771136in}{3.098189in}}%
\pgfpathlineto{\pgfqpoint{4.869773in}{3.056216in}}%
\pgfpathlineto{\pgfqpoint{4.968409in}{3.224333in}}%
\pgfpathlineto{\pgfqpoint{5.067045in}{3.129425in}}%
\pgfpathlineto{\pgfqpoint{5.165682in}{3.116517in}}%
\pgfpathlineto{\pgfqpoint{5.264318in}{3.320058in}}%
\pgfpathlineto{\pgfqpoint{5.362955in}{3.225176in}}%
\pgfpathlineto{\pgfqpoint{5.461591in}{2.956912in}}%
\pgfpathlineto{\pgfqpoint{5.560227in}{3.147304in}}%
\pgfpathlineto{\pgfqpoint{5.658864in}{3.196037in}}%
\pgfpathlineto{\pgfqpoint{5.757500in}{2.978027in}}%
\pgfpathlineto{\pgfqpoint{5.856136in}{2.812910in}}%
\pgfpathlineto{\pgfqpoint{5.954773in}{3.295488in}}%
\pgfpathlineto{\pgfqpoint{6.053409in}{3.192202in}}%
\pgfusepath{stroke}%
\end{pgfscope}%
\begin{pgfscope}%
\pgfsetrectcap%
\pgfsetmiterjoin%
\pgfsetlinewidth{0.803000pt}%
\definecolor{currentstroke}{rgb}{0.000000,0.000000,0.000000}%
\pgfsetstrokecolor{currentstroke}%
\pgfsetdash{}{0pt}%
\pgfpathmoveto{\pgfqpoint{0.875000in}{0.440000in}}%
\pgfpathlineto{\pgfqpoint{0.875000in}{3.520000in}}%
\pgfusepath{stroke}%
\end{pgfscope}%
\begin{pgfscope}%
\pgfsetrectcap%
\pgfsetmiterjoin%
\pgfsetlinewidth{0.803000pt}%
\definecolor{currentstroke}{rgb}{0.000000,0.000000,0.000000}%
\pgfsetstrokecolor{currentstroke}%
\pgfsetdash{}{0pt}%
\pgfpathmoveto{\pgfqpoint{6.300000in}{0.440000in}}%
\pgfpathlineto{\pgfqpoint{6.300000in}{3.520000in}}%
\pgfusepath{stroke}%
\end{pgfscope}%
\begin{pgfscope}%
\pgfsetrectcap%
\pgfsetmiterjoin%
\pgfsetlinewidth{0.803000pt}%
\definecolor{currentstroke}{rgb}{0.000000,0.000000,0.000000}%
\pgfsetstrokecolor{currentstroke}%
\pgfsetdash{}{0pt}%
\pgfpathmoveto{\pgfqpoint{0.875000in}{0.440000in}}%
\pgfpathlineto{\pgfqpoint{6.300000in}{0.440000in}}%
\pgfusepath{stroke}%
\end{pgfscope}%
\begin{pgfscope}%
\pgfsetrectcap%
\pgfsetmiterjoin%
\pgfsetlinewidth{0.803000pt}%
\definecolor{currentstroke}{rgb}{0.000000,0.000000,0.000000}%
\pgfsetstrokecolor{currentstroke}%
\pgfsetdash{}{0pt}%
\pgfpathmoveto{\pgfqpoint{0.875000in}{3.520000in}}%
\pgfpathlineto{\pgfqpoint{6.300000in}{3.520000in}}%
\pgfusepath{stroke}%
\end{pgfscope}%
\end{pgfpicture}%
\makeatother%
\endgroup%
}
    \caption[Accuracy on each epoch for AFRICAN Pretraining]{This chart illustrates the epoch-wise performance of AFRICAN pretraining.}
    \label{fig:tp_africanpretraining}
\end{figure}

To delve deeper, I draw attention maps for the vision transformer within the model. Thanks to the transformer's query-key-value mechanism, these maps can be computed by accumulating the connection weights between the keys and queries across all layers. This approach helps highlight the most frequently queried keys and identify the patches that heavily influence the final output. The attention maps using this method on different videos are shown in Figures \ref{fig:attnmap1} and \ref{fig:attnmap2}. Clearly, the attention map from the with-AFRICAN model is more adept at pinpointing the animal and even its boundary compared to the "w/o-AFRICAN" model.

\begin{figure}[ht]
    \centering
    \includegraphics[width=1.0\textwidth]{assets/imgs/5_4_AttentionMaps_1}
    \caption[Attention Map 1]{The attention map for three videos. The first row is the input video, the second row is the attention map of the w/o-AFRICAN model, and the third row is the attention map of the with-AFRICAN model.}
    \label{fig:attnmap1}
\end{figure}

\begin{figure}[ht]
    \centering
    \includegraphics[width=1.0\textwidth]{assets/imgs/5_4_AttentionMaps_2}
    \caption[Attention Map 1]{The same with Figure \ref{fig:attnmap1} for another three videos.}
    \label{fig:attnmap2}
\end{figure}







\subsection{AFRICAN for Action Recognition}
The AFRICAN model for action recognition denoted as AFRICAN-AR, the structure is illustrated in Figure \ref{fig:modelstructaf_ar}, which is trained with a binary cross-entropy loss and adamw optimizer, using a learning rate of 0.00015 on a single A100 GPU. Since the Image CLIP model converged earlier than Video CLIP based model, as illustrated in Figure \ref{fig:tp_backbone}where the IC model does not improve after Epoch 40, I train the model for 40 epochs for the experiment. The mAP is also used as the evaluation metric in this task.

The results for action recognition at Epoch 40 are presented in Table \ref{tab:allresults40}, and the results at the best epoch are shown in Table \ref{tab:allresultsbest}. With the stream of AFRICAN pretrained weights, the AFRICAN-AR are able to outperforms the other models in all measured aspects at Epoch 40, with an overall mean Average Precision (mAP) of 54.02\%. At the best epoch, the AFRICAN-AR model generally performs best with an overall mAP of 55.08\%, except in the Tail metric, where IC takes the lead with 48.96\%.


\begin{table}[h]
    \centering
    \caption{Results of action recognition (Epoch 40)}
    \label{tab:allresults40}
    \begin{tabular}{lllll}
        \toprule
        \multirow{2}{*}{Models} & \multicolumn{4}{c}{mAP} \\
        \cmidrule{2-5} 
        {} & Overall & Head  & Middle & Tail \\
        \midrule
        CARe          & 30.55   & 63.33 & 38.62 & 25.09 \\
        IC            & 52.33   & 69.53 & 60.52 & 45.79 \\        
        AFRICAN-AR    & \textbf{54.02} & \textbf{73.54} & \textbf{64.38} & \textbf{46.21} \\
        \bottomrule
    \end{tabular}
\end{table}

\begin{table}[h]
    \centering
    \caption{Results of action recognition (Best Epoch)}
    \label{tab:allresultsbest}
    \begin{tabular}{lllll}
        \toprule
        \multirow{2}{*}{Models} & \multicolumn{4}{c}{mAP} \\
        \cmidrule{2-5} 
        {} & Overall & Head  & Middle & Tail \\
        \midrule
        CARe        & 30.55   & 63.33 & 38.62  & 25.09 \\
        IC          & 54.67   & 71.72 & 63.31 & \textbf{48.96} \\
        AFRICAN     & \textbf{55.08} & \textbf{74.16} & \textbf{65.60} & 47.75 \\
        \bottomrule
    \end{tabular}
\end{table}






% % Need ICs1_B
% % Need ICAFs1_B
% % Need ICs1loh B (failed)
% % Need ICs1loh F
% % Need VCs1dd_bs008 F
% % Need VCs1dd_bs032 F
% % Need VCs1_B no text embedding proj


% TODO
% 0. evaluate on the best score
% 1. Add VC vs. IC Section of Ablation Study (Further training 2 epochs for VCdd)
% 2. Update Table for Backbone Selection (wait ICs1_B)
% 3. Update Table For AFRICAN-AR (wait ICAFs1_B)
% 4. (Time Permmited) Add FocalLoss of IC Section Ablation Study
% 4. (Time Permmited) Add batch size section of Ablation Study (VCs1dd_bs008_B and VCs1dd_bs008_B)

